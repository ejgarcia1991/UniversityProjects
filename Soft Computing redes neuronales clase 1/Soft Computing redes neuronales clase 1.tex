\documentclass[11pt]{article}

    \usepackage[breakable]{tcolorbox}
    \usepackage{parskip} % Stop auto-indenting (to mimic markdown behaviour)
    
    \usepackage{iftex}
    \ifPDFTeX
    	\usepackage[T1]{fontenc}
    	\usepackage{mathpazo}
    \else
    	\usepackage{fontspec}
    \fi

    % Basic figure setup, for now with no caption control since it's done
    % automatically by Pandoc (which extracts ![](path) syntax from Markdown).
    \usepackage{graphicx}
    % Maintain compatibility with old templates. Remove in nbconvert 6.0
    \let\Oldincludegraphics\includegraphics
    % Ensure that by default, figures have no caption (until we provide a
    % proper Figure object with a Caption API and a way to capture that
    % in the conversion process - todo).
    \usepackage{caption}
    \DeclareCaptionFormat{nocaption}{}
    \captionsetup{format=nocaption,aboveskip=0pt,belowskip=0pt}

    \usepackage[Export]{adjustbox} % Used to constrain images to a maximum size
    \adjustboxset{max size={0.9\linewidth}{0.9\paperheight}}
    \usepackage{float}
    \floatplacement{figure}{H} % forces figures to be placed at the correct location
    \usepackage{xcolor} % Allow colors to be defined
    \usepackage{enumerate} % Needed for markdown enumerations to work
    \usepackage{geometry} % Used to adjust the document margins
    \usepackage{amsmath} % Equations
    \usepackage{amssymb} % Equations
    \usepackage{textcomp} % defines textquotesingle
    % Hack from http://tex.stackexchange.com/a/47451/13684:
    \AtBeginDocument{%
        \def\PYZsq{\textquotesingle}% Upright quotes in Pygmentized code
    }
    \usepackage{upquote} % Upright quotes for verbatim code
    \usepackage{eurosym} % defines \euro
    \usepackage[mathletters]{ucs} % Extended unicode (utf-8) support
    \usepackage{fancyvrb} % verbatim replacement that allows latex
    \usepackage{grffile} % extends the file name processing of package graphics 
                         % to support a larger range
    \makeatletter % fix for grffile with XeLaTeX
    \def\Gread@@xetex#1{%
      \IfFileExists{"\Gin@base".bb}%
      {\Gread@eps{\Gin@base.bb}}%
      {\Gread@@xetex@aux#1}%
    }
    \makeatother

    % The hyperref package gives us a pdf with properly built
    % internal navigation ('pdf bookmarks' for the table of contents,
    % internal cross-reference links, web links for URLs, etc.)
    \usepackage{hyperref}
    % The default LaTeX title has an obnoxious amount of whitespace. By default,
    % titling removes some of it. It also provides customization options.
    \usepackage{titling}
    \usepackage{longtable} % longtable support required by pandoc >1.10
    \usepackage{booktabs}  % table support for pandoc > 1.12.2
    \usepackage[inline]{enumitem} % IRkernel/repr support (it uses the enumerate* environment)
    \usepackage[normalem]{ulem} % ulem is needed to support strikethroughs (\sout)
                                % normalem makes italics be italics, not underlines
    \usepackage{mathrsfs}
    

    
    % Colors for the hyperref package
    \definecolor{urlcolor}{rgb}{0,.145,.698}
    \definecolor{linkcolor}{rgb}{.71,0.21,0.01}
    \definecolor{citecolor}{rgb}{.12,.54,.11}

    % ANSI colors
    \definecolor{ansi-black}{HTML}{3E424D}
    \definecolor{ansi-black-intense}{HTML}{282C36}
    \definecolor{ansi-red}{HTML}{E75C58}
    \definecolor{ansi-red-intense}{HTML}{B22B31}
    \definecolor{ansi-green}{HTML}{00A250}
    \definecolor{ansi-green-intense}{HTML}{007427}
    \definecolor{ansi-yellow}{HTML}{DDB62B}
    \definecolor{ansi-yellow-intense}{HTML}{B27D12}
    \definecolor{ansi-blue}{HTML}{208FFB}
    \definecolor{ansi-blue-intense}{HTML}{0065CA}
    \definecolor{ansi-magenta}{HTML}{D160C4}
    \definecolor{ansi-magenta-intense}{HTML}{A03196}
    \definecolor{ansi-cyan}{HTML}{60C6C8}
    \definecolor{ansi-cyan-intense}{HTML}{258F8F}
    \definecolor{ansi-white}{HTML}{C5C1B4}
    \definecolor{ansi-white-intense}{HTML}{A1A6B2}
    \definecolor{ansi-default-inverse-fg}{HTML}{FFFFFF}
    \definecolor{ansi-default-inverse-bg}{HTML}{000000}

    % commands and environments needed by pandoc snippets
    % extracted from the output of `pandoc -s`
    \providecommand{\tightlist}{%
      \setlength{\itemsep}{0pt}\setlength{\parskip}{0pt}}
    \DefineVerbatimEnvironment{Highlighting}{Verbatim}{commandchars=\\\{\}}
    % Add ',fontsize=\small' for more characters per line
    \newenvironment{Shaded}{}{}
    \newcommand{\KeywordTok}[1]{\textcolor[rgb]{0.00,0.44,0.13}{\textbf{{#1}}}}
    \newcommand{\DataTypeTok}[1]{\textcolor[rgb]{0.56,0.13,0.00}{{#1}}}
    \newcommand{\DecValTok}[1]{\textcolor[rgb]{0.25,0.63,0.44}{{#1}}}
    \newcommand{\BaseNTok}[1]{\textcolor[rgb]{0.25,0.63,0.44}{{#1}}}
    \newcommand{\FloatTok}[1]{\textcolor[rgb]{0.25,0.63,0.44}{{#1}}}
    \newcommand{\CharTok}[1]{\textcolor[rgb]{0.25,0.44,0.63}{{#1}}}
    \newcommand{\StringTok}[1]{\textcolor[rgb]{0.25,0.44,0.63}{{#1}}}
    \newcommand{\CommentTok}[1]{\textcolor[rgb]{0.38,0.63,0.69}{\textit{{#1}}}}
    \newcommand{\OtherTok}[1]{\textcolor[rgb]{0.00,0.44,0.13}{{#1}}}
    \newcommand{\AlertTok}[1]{\textcolor[rgb]{1.00,0.00,0.00}{\textbf{{#1}}}}
    \newcommand{\FunctionTok}[1]{\textcolor[rgb]{0.02,0.16,0.49}{{#1}}}
    \newcommand{\RegionMarkerTok}[1]{{#1}}
    \newcommand{\ErrorTok}[1]{\textcolor[rgb]{1.00,0.00,0.00}{\textbf{{#1}}}}
    \newcommand{\NormalTok}[1]{{#1}}
    
    % Additional commands for more recent versions of Pandoc
    \newcommand{\ConstantTok}[1]{\textcolor[rgb]{0.53,0.00,0.00}{{#1}}}
    \newcommand{\SpecialCharTok}[1]{\textcolor[rgb]{0.25,0.44,0.63}{{#1}}}
    \newcommand{\VerbatimStringTok}[1]{\textcolor[rgb]{0.25,0.44,0.63}{{#1}}}
    \newcommand{\SpecialStringTok}[1]{\textcolor[rgb]{0.73,0.40,0.53}{{#1}}}
    \newcommand{\ImportTok}[1]{{#1}}
    \newcommand{\DocumentationTok}[1]{\textcolor[rgb]{0.73,0.13,0.13}{\textit{{#1}}}}
    \newcommand{\AnnotationTok}[1]{\textcolor[rgb]{0.38,0.63,0.69}{\textbf{\textit{{#1}}}}}
    \newcommand{\CommentVarTok}[1]{\textcolor[rgb]{0.38,0.63,0.69}{\textbf{\textit{{#1}}}}}
    \newcommand{\VariableTok}[1]{\textcolor[rgb]{0.10,0.09,0.49}{{#1}}}
    \newcommand{\ControlFlowTok}[1]{\textcolor[rgb]{0.00,0.44,0.13}{\textbf{{#1}}}}
    \newcommand{\OperatorTok}[1]{\textcolor[rgb]{0.40,0.40,0.40}{{#1}}}
    \newcommand{\BuiltInTok}[1]{{#1}}
    \newcommand{\ExtensionTok}[1]{{#1}}
    \newcommand{\PreprocessorTok}[1]{\textcolor[rgb]{0.74,0.48,0.00}{{#1}}}
    \newcommand{\AttributeTok}[1]{\textcolor[rgb]{0.49,0.56,0.16}{{#1}}}
    \newcommand{\InformationTok}[1]{\textcolor[rgb]{0.38,0.63,0.69}{\textbf{\textit{{#1}}}}}
    \newcommand{\WarningTok}[1]{\textcolor[rgb]{0.38,0.63,0.69}{\textbf{\textit{{#1}}}}}
    
    
    % Define a nice break command that doesn't care if a line doesn't already
    % exist.
    \def\br{\hspace*{\fill} \\* }
    % Math Jax compatibility definitions
    \def\gt{>}
    \def\lt{<}
    \let\Oldtex\TeX
    \let\Oldlatex\LaTeX
    \renewcommand{\TeX}{\textrm{\Oldtex}}
    \renewcommand{\LaTeX}{\textrm{\Oldlatex}}
    % Document parameters
    % Document title
    \title{Soft Computing redes neuronales clase 1}
    
    
    
    
    
% Pygments definitions
\makeatletter
\def\PY@reset{\let\PY@it=\relax \let\PY@bf=\relax%
    \let\PY@ul=\relax \let\PY@tc=\relax%
    \let\PY@bc=\relax \let\PY@ff=\relax}
\def\PY@tok#1{\csname PY@tok@#1\endcsname}
\def\PY@toks#1+{\ifx\relax#1\empty\else%
    \PY@tok{#1}\expandafter\PY@toks\fi}
\def\PY@do#1{\PY@bc{\PY@tc{\PY@ul{%
    \PY@it{\PY@bf{\PY@ff{#1}}}}}}}
\def\PY#1#2{\PY@reset\PY@toks#1+\relax+\PY@do{#2}}

\expandafter\def\csname PY@tok@w\endcsname{\def\PY@tc##1{\textcolor[rgb]{0.73,0.73,0.73}{##1}}}
\expandafter\def\csname PY@tok@c\endcsname{\let\PY@it=\textit\def\PY@tc##1{\textcolor[rgb]{0.25,0.50,0.50}{##1}}}
\expandafter\def\csname PY@tok@cp\endcsname{\def\PY@tc##1{\textcolor[rgb]{0.74,0.48,0.00}{##1}}}
\expandafter\def\csname PY@tok@k\endcsname{\let\PY@bf=\textbf\def\PY@tc##1{\textcolor[rgb]{0.00,0.50,0.00}{##1}}}
\expandafter\def\csname PY@tok@kp\endcsname{\def\PY@tc##1{\textcolor[rgb]{0.00,0.50,0.00}{##1}}}
\expandafter\def\csname PY@tok@kt\endcsname{\def\PY@tc##1{\textcolor[rgb]{0.69,0.00,0.25}{##1}}}
\expandafter\def\csname PY@tok@o\endcsname{\def\PY@tc##1{\textcolor[rgb]{0.40,0.40,0.40}{##1}}}
\expandafter\def\csname PY@tok@ow\endcsname{\let\PY@bf=\textbf\def\PY@tc##1{\textcolor[rgb]{0.67,0.13,1.00}{##1}}}
\expandafter\def\csname PY@tok@nb\endcsname{\def\PY@tc##1{\textcolor[rgb]{0.00,0.50,0.00}{##1}}}
\expandafter\def\csname PY@tok@nf\endcsname{\def\PY@tc##1{\textcolor[rgb]{0.00,0.00,1.00}{##1}}}
\expandafter\def\csname PY@tok@nc\endcsname{\let\PY@bf=\textbf\def\PY@tc##1{\textcolor[rgb]{0.00,0.00,1.00}{##1}}}
\expandafter\def\csname PY@tok@nn\endcsname{\let\PY@bf=\textbf\def\PY@tc##1{\textcolor[rgb]{0.00,0.00,1.00}{##1}}}
\expandafter\def\csname PY@tok@ne\endcsname{\let\PY@bf=\textbf\def\PY@tc##1{\textcolor[rgb]{0.82,0.25,0.23}{##1}}}
\expandafter\def\csname PY@tok@nv\endcsname{\def\PY@tc##1{\textcolor[rgb]{0.10,0.09,0.49}{##1}}}
\expandafter\def\csname PY@tok@no\endcsname{\def\PY@tc##1{\textcolor[rgb]{0.53,0.00,0.00}{##1}}}
\expandafter\def\csname PY@tok@nl\endcsname{\def\PY@tc##1{\textcolor[rgb]{0.63,0.63,0.00}{##1}}}
\expandafter\def\csname PY@tok@ni\endcsname{\let\PY@bf=\textbf\def\PY@tc##1{\textcolor[rgb]{0.60,0.60,0.60}{##1}}}
\expandafter\def\csname PY@tok@na\endcsname{\def\PY@tc##1{\textcolor[rgb]{0.49,0.56,0.16}{##1}}}
\expandafter\def\csname PY@tok@nt\endcsname{\let\PY@bf=\textbf\def\PY@tc##1{\textcolor[rgb]{0.00,0.50,0.00}{##1}}}
\expandafter\def\csname PY@tok@nd\endcsname{\def\PY@tc##1{\textcolor[rgb]{0.67,0.13,1.00}{##1}}}
\expandafter\def\csname PY@tok@s\endcsname{\def\PY@tc##1{\textcolor[rgb]{0.73,0.13,0.13}{##1}}}
\expandafter\def\csname PY@tok@sd\endcsname{\let\PY@it=\textit\def\PY@tc##1{\textcolor[rgb]{0.73,0.13,0.13}{##1}}}
\expandafter\def\csname PY@tok@si\endcsname{\let\PY@bf=\textbf\def\PY@tc##1{\textcolor[rgb]{0.73,0.40,0.53}{##1}}}
\expandafter\def\csname PY@tok@se\endcsname{\let\PY@bf=\textbf\def\PY@tc##1{\textcolor[rgb]{0.73,0.40,0.13}{##1}}}
\expandafter\def\csname PY@tok@sr\endcsname{\def\PY@tc##1{\textcolor[rgb]{0.73,0.40,0.53}{##1}}}
\expandafter\def\csname PY@tok@ss\endcsname{\def\PY@tc##1{\textcolor[rgb]{0.10,0.09,0.49}{##1}}}
\expandafter\def\csname PY@tok@sx\endcsname{\def\PY@tc##1{\textcolor[rgb]{0.00,0.50,0.00}{##1}}}
\expandafter\def\csname PY@tok@m\endcsname{\def\PY@tc##1{\textcolor[rgb]{0.40,0.40,0.40}{##1}}}
\expandafter\def\csname PY@tok@gh\endcsname{\let\PY@bf=\textbf\def\PY@tc##1{\textcolor[rgb]{0.00,0.00,0.50}{##1}}}
\expandafter\def\csname PY@tok@gu\endcsname{\let\PY@bf=\textbf\def\PY@tc##1{\textcolor[rgb]{0.50,0.00,0.50}{##1}}}
\expandafter\def\csname PY@tok@gd\endcsname{\def\PY@tc##1{\textcolor[rgb]{0.63,0.00,0.00}{##1}}}
\expandafter\def\csname PY@tok@gi\endcsname{\def\PY@tc##1{\textcolor[rgb]{0.00,0.63,0.00}{##1}}}
\expandafter\def\csname PY@tok@gr\endcsname{\def\PY@tc##1{\textcolor[rgb]{1.00,0.00,0.00}{##1}}}
\expandafter\def\csname PY@tok@ge\endcsname{\let\PY@it=\textit}
\expandafter\def\csname PY@tok@gs\endcsname{\let\PY@bf=\textbf}
\expandafter\def\csname PY@tok@gp\endcsname{\let\PY@bf=\textbf\def\PY@tc##1{\textcolor[rgb]{0.00,0.00,0.50}{##1}}}
\expandafter\def\csname PY@tok@go\endcsname{\def\PY@tc##1{\textcolor[rgb]{0.53,0.53,0.53}{##1}}}
\expandafter\def\csname PY@tok@gt\endcsname{\def\PY@tc##1{\textcolor[rgb]{0.00,0.27,0.87}{##1}}}
\expandafter\def\csname PY@tok@err\endcsname{\def\PY@bc##1{\setlength{\fboxsep}{0pt}\fcolorbox[rgb]{1.00,0.00,0.00}{1,1,1}{\strut ##1}}}
\expandafter\def\csname PY@tok@kc\endcsname{\let\PY@bf=\textbf\def\PY@tc##1{\textcolor[rgb]{0.00,0.50,0.00}{##1}}}
\expandafter\def\csname PY@tok@kd\endcsname{\let\PY@bf=\textbf\def\PY@tc##1{\textcolor[rgb]{0.00,0.50,0.00}{##1}}}
\expandafter\def\csname PY@tok@kn\endcsname{\let\PY@bf=\textbf\def\PY@tc##1{\textcolor[rgb]{0.00,0.50,0.00}{##1}}}
\expandafter\def\csname PY@tok@kr\endcsname{\let\PY@bf=\textbf\def\PY@tc##1{\textcolor[rgb]{0.00,0.50,0.00}{##1}}}
\expandafter\def\csname PY@tok@bp\endcsname{\def\PY@tc##1{\textcolor[rgb]{0.00,0.50,0.00}{##1}}}
\expandafter\def\csname PY@tok@fm\endcsname{\def\PY@tc##1{\textcolor[rgb]{0.00,0.00,1.00}{##1}}}
\expandafter\def\csname PY@tok@vc\endcsname{\def\PY@tc##1{\textcolor[rgb]{0.10,0.09,0.49}{##1}}}
\expandafter\def\csname PY@tok@vg\endcsname{\def\PY@tc##1{\textcolor[rgb]{0.10,0.09,0.49}{##1}}}
\expandafter\def\csname PY@tok@vi\endcsname{\def\PY@tc##1{\textcolor[rgb]{0.10,0.09,0.49}{##1}}}
\expandafter\def\csname PY@tok@vm\endcsname{\def\PY@tc##1{\textcolor[rgb]{0.10,0.09,0.49}{##1}}}
\expandafter\def\csname PY@tok@sa\endcsname{\def\PY@tc##1{\textcolor[rgb]{0.73,0.13,0.13}{##1}}}
\expandafter\def\csname PY@tok@sb\endcsname{\def\PY@tc##1{\textcolor[rgb]{0.73,0.13,0.13}{##1}}}
\expandafter\def\csname PY@tok@sc\endcsname{\def\PY@tc##1{\textcolor[rgb]{0.73,0.13,0.13}{##1}}}
\expandafter\def\csname PY@tok@dl\endcsname{\def\PY@tc##1{\textcolor[rgb]{0.73,0.13,0.13}{##1}}}
\expandafter\def\csname PY@tok@s2\endcsname{\def\PY@tc##1{\textcolor[rgb]{0.73,0.13,0.13}{##1}}}
\expandafter\def\csname PY@tok@sh\endcsname{\def\PY@tc##1{\textcolor[rgb]{0.73,0.13,0.13}{##1}}}
\expandafter\def\csname PY@tok@s1\endcsname{\def\PY@tc##1{\textcolor[rgb]{0.73,0.13,0.13}{##1}}}
\expandafter\def\csname PY@tok@mb\endcsname{\def\PY@tc##1{\textcolor[rgb]{0.40,0.40,0.40}{##1}}}
\expandafter\def\csname PY@tok@mf\endcsname{\def\PY@tc##1{\textcolor[rgb]{0.40,0.40,0.40}{##1}}}
\expandafter\def\csname PY@tok@mh\endcsname{\def\PY@tc##1{\textcolor[rgb]{0.40,0.40,0.40}{##1}}}
\expandafter\def\csname PY@tok@mi\endcsname{\def\PY@tc##1{\textcolor[rgb]{0.40,0.40,0.40}{##1}}}
\expandafter\def\csname PY@tok@il\endcsname{\def\PY@tc##1{\textcolor[rgb]{0.40,0.40,0.40}{##1}}}
\expandafter\def\csname PY@tok@mo\endcsname{\def\PY@tc##1{\textcolor[rgb]{0.40,0.40,0.40}{##1}}}
\expandafter\def\csname PY@tok@ch\endcsname{\let\PY@it=\textit\def\PY@tc##1{\textcolor[rgb]{0.25,0.50,0.50}{##1}}}
\expandafter\def\csname PY@tok@cm\endcsname{\let\PY@it=\textit\def\PY@tc##1{\textcolor[rgb]{0.25,0.50,0.50}{##1}}}
\expandafter\def\csname PY@tok@cpf\endcsname{\let\PY@it=\textit\def\PY@tc##1{\textcolor[rgb]{0.25,0.50,0.50}{##1}}}
\expandafter\def\csname PY@tok@c1\endcsname{\let\PY@it=\textit\def\PY@tc##1{\textcolor[rgb]{0.25,0.50,0.50}{##1}}}
\expandafter\def\csname PY@tok@cs\endcsname{\let\PY@it=\textit\def\PY@tc##1{\textcolor[rgb]{0.25,0.50,0.50}{##1}}}

\def\PYZbs{\char`\\}
\def\PYZus{\char`\_}
\def\PYZob{\char`\{}
\def\PYZcb{\char`\}}
\def\PYZca{\char`\^}
\def\PYZam{\char`\&}
\def\PYZlt{\char`\<}
\def\PYZgt{\char`\>}
\def\PYZsh{\char`\#}
\def\PYZpc{\char`\%}
\def\PYZdl{\char`\$}
\def\PYZhy{\char`\-}
\def\PYZsq{\char`\'}
\def\PYZdq{\char`\"}
\def\PYZti{\char`\~}
% for compatibility with earlier versions
\def\PYZat{@}
\def\PYZlb{[}
\def\PYZrb{]}
\makeatother


    % For linebreaks inside Verbatim environment from package fancyvrb. 
    \makeatletter
        \newbox\Wrappedcontinuationbox 
        \newbox\Wrappedvisiblespacebox 
        \newcommand*\Wrappedvisiblespace {\textcolor{red}{\textvisiblespace}} 
        \newcommand*\Wrappedcontinuationsymbol {\textcolor{red}{\llap{\tiny$\m@th\hookrightarrow$}}} 
        \newcommand*\Wrappedcontinuationindent {3ex } 
        \newcommand*\Wrappedafterbreak {\kern\Wrappedcontinuationindent\copy\Wrappedcontinuationbox} 
        % Take advantage of the already applied Pygments mark-up to insert 
        % potential linebreaks for TeX processing. 
        %        {, <, #, %, $, ' and ": go to next line. 
        %        _, }, ^, &, >, - and ~: stay at end of broken line. 
        % Use of \textquotesingle for straight quote. 
        \newcommand*\Wrappedbreaksatspecials {% 
            \def\PYGZus{\discretionary{\char`\_}{\Wrappedafterbreak}{\char`\_}}% 
            \def\PYGZob{\discretionary{}{\Wrappedafterbreak\char`\{}{\char`\{}}% 
            \def\PYGZcb{\discretionary{\char`\}}{\Wrappedafterbreak}{\char`\}}}% 
            \def\PYGZca{\discretionary{\char`\^}{\Wrappedafterbreak}{\char`\^}}% 
            \def\PYGZam{\discretionary{\char`\&}{\Wrappedafterbreak}{\char`\&}}% 
            \def\PYGZlt{\discretionary{}{\Wrappedafterbreak\char`\<}{\char`\<}}% 
            \def\PYGZgt{\discretionary{\char`\>}{\Wrappedafterbreak}{\char`\>}}% 
            \def\PYGZsh{\discretionary{}{\Wrappedafterbreak\char`\#}{\char`\#}}% 
            \def\PYGZpc{\discretionary{}{\Wrappedafterbreak\char`\%}{\char`\%}}% 
            \def\PYGZdl{\discretionary{}{\Wrappedafterbreak\char`\$}{\char`\$}}% 
            \def\PYGZhy{\discretionary{\char`\-}{\Wrappedafterbreak}{\char`\-}}% 
            \def\PYGZsq{\discretionary{}{\Wrappedafterbreak\textquotesingle}{\textquotesingle}}% 
            \def\PYGZdq{\discretionary{}{\Wrappedafterbreak\char`\"}{\char`\"}}% 
            \def\PYGZti{\discretionary{\char`\~}{\Wrappedafterbreak}{\char`\~}}% 
        } 
        % Some characters . , ; ? ! / are not pygmentized. 
        % This macro makes them "active" and they will insert potential linebreaks 
        \newcommand*\Wrappedbreaksatpunct {% 
            \lccode`\~`\.\lowercase{\def~}{\discretionary{\hbox{\char`\.}}{\Wrappedafterbreak}{\hbox{\char`\.}}}% 
            \lccode`\~`\,\lowercase{\def~}{\discretionary{\hbox{\char`\,}}{\Wrappedafterbreak}{\hbox{\char`\,}}}% 
            \lccode`\~`\;\lowercase{\def~}{\discretionary{\hbox{\char`\;}}{\Wrappedafterbreak}{\hbox{\char`\;}}}% 
            \lccode`\~`\:\lowercase{\def~}{\discretionary{\hbox{\char`\:}}{\Wrappedafterbreak}{\hbox{\char`\:}}}% 
            \lccode`\~`\?\lowercase{\def~}{\discretionary{\hbox{\char`\?}}{\Wrappedafterbreak}{\hbox{\char`\?}}}% 
            \lccode`\~`\!\lowercase{\def~}{\discretionary{\hbox{\char`\!}}{\Wrappedafterbreak}{\hbox{\char`\!}}}% 
            \lccode`\~`\/\lowercase{\def~}{\discretionary{\hbox{\char`\/}}{\Wrappedafterbreak}{\hbox{\char`\/}}}% 
            \catcode`\.\active
            \catcode`\,\active 
            \catcode`\;\active
            \catcode`\:\active
            \catcode`\?\active
            \catcode`\!\active
            \catcode`\/\active 
            \lccode`\~`\~ 	
        }
    \makeatother

    \let\OriginalVerbatim=\Verbatim
    \makeatletter
    \renewcommand{\Verbatim}[1][1]{%
        %\parskip\z@skip
        \sbox\Wrappedcontinuationbox {\Wrappedcontinuationsymbol}%
        \sbox\Wrappedvisiblespacebox {\FV@SetupFont\Wrappedvisiblespace}%
        \def\FancyVerbFormatLine ##1{\hsize\linewidth
            \vtop{\raggedright\hyphenpenalty\z@\exhyphenpenalty\z@
                \doublehyphendemerits\z@\finalhyphendemerits\z@
                \strut ##1\strut}%
        }%
        % If the linebreak is at a space, the latter will be displayed as visible
        % space at end of first line, and a continuation symbol starts next line.
        % Stretch/shrink are however usually zero for typewriter font.
        \def\FV@Space {%
            \nobreak\hskip\z@ plus\fontdimen3\font minus\fontdimen4\font
            \discretionary{\copy\Wrappedvisiblespacebox}{\Wrappedafterbreak}
            {\kern\fontdimen2\font}%
        }%
        
        % Allow breaks at special characters using \PYG... macros.
        \Wrappedbreaksatspecials
        % Breaks at punctuation characters . , ; ? ! and / need catcode=\active 	
        \OriginalVerbatim[#1,codes*=\Wrappedbreaksatpunct]%
    }
    \makeatother

    % Exact colors from NB
    \definecolor{incolor}{HTML}{303F9F}
    \definecolor{outcolor}{HTML}{D84315}
    \definecolor{cellborder}{HTML}{CFCFCF}
    \definecolor{cellbackground}{HTML}{F7F7F7}
    
    % prompt
    \makeatletter
    \newcommand{\boxspacing}{\kern\kvtcb@left@rule\kern\kvtcb@boxsep}
    \makeatother
    \newcommand{\prompt}[4]{
        \ttfamily\llap{{\color{#2}[#3]:\hspace{3pt}#4}}\vspace{-\baselineskip}
    }
    

    
    % Prevent overflowing lines due to hard-to-break entities
    \sloppy 
    % Setup hyperref package
    \hypersetup{
      breaklinks=true,  % so long urls are correctly broken across lines
      colorlinks=true,
      urlcolor=urlcolor,
      linkcolor=linkcolor,
      citecolor=citecolor,
      }
    % Slightly bigger margins than the latex defaults
    
    \geometry{verbose,tmargin=1in,bmargin=1in,lmargin=1in,rmargin=1in}
    
    

\begin{document}
    
    \maketitle
    
    

    
    We load data here, and confirm some basic data and see if it loaded
correctly and it's complete

    \begin{tcolorbox}[breakable, size=fbox, boxrule=1pt, pad at break*=1mm,colback=cellbackground, colframe=cellborder]
\prompt{In}{incolor}{1}{\boxspacing}
\begin{Verbatim}[commandchars=\\\{\}]
\PY{k+kn}{import} \PY{n+nn}{pandas} \PY{k}{as} \PY{n+nn}{pd}
\PY{n}{white\PYZus{}wine} \PY{o}{=} \PY{n}{pd}\PY{o}{.}\PY{n}{read\PYZus{}csv}\PY{p}{(}\PY{l+s+s1}{\PYZsq{}}\PY{l+s+s1}{winequality\PYZhy{}white.csv}\PY{l+s+s1}{\PYZsq{}}\PY{p}{,}\PY{n}{sep}\PY{o}{=}\PY{l+s+s1}{\PYZsq{}}\PY{l+s+s1}{;}\PY{l+s+s1}{\PYZsq{}}\PY{p}{)}
\PY{n}{red\PYZus{}wine}\PY{o}{=} \PY{n}{pd}\PY{o}{.}\PY{n}{read\PYZus{}csv}\PY{p}{(}\PY{l+s+s1}{\PYZsq{}}\PY{l+s+s1}{winequality\PYZhy{}red.csv}\PY{l+s+s1}{\PYZsq{}}\PY{p}{,}\PY{n}{sep}\PY{o}{=}\PY{l+s+s1}{\PYZsq{}}\PY{l+s+s1}{;}\PY{l+s+s1}{\PYZsq{}}\PY{p}{)}
\end{Verbatim}
\end{tcolorbox}

    \begin{tcolorbox}[breakable, size=fbox, boxrule=1pt, pad at break*=1mm,colback=cellbackground, colframe=cellborder]
\prompt{In}{incolor}{2}{\boxspacing}
\begin{Verbatim}[commandchars=\\\{\}]
\PY{n+nb}{print}\PY{p}{(}\PY{n}{white\PYZus{}wine}\PY{o}{.}\PY{n}{info}\PY{p}{)}
\end{Verbatim}
\end{tcolorbox}

    \begin{Verbatim}[commandchars=\\\{\}]
<bound method DataFrame.info of       fixed acidity  volatile acidity  citric
acid  residual sugar  chlorides  \textbackslash{}
0               7.0              0.27         0.36            20.7      0.045
1               6.3              0.30         0.34             1.6      0.049
2               8.1              0.28         0.40             6.9      0.050
3               7.2              0.23         0.32             8.5      0.058
4               7.2              0.23         0.32             8.5      0.058
{\ldots}             {\ldots}               {\ldots}          {\ldots}             {\ldots}        {\ldots}
4893            6.2              0.21         0.29             1.6      0.039
4894            6.6              0.32         0.36             8.0      0.047
4895            6.5              0.24         0.19             1.2      0.041
4896            5.5              0.29         0.30             1.1      0.022
4897            6.0              0.21         0.38             0.8      0.020

      free sulfur dioxide  total sulfur dioxide  density    pH  sulphates  \textbackslash{}
0                    45.0                 170.0  1.00100  3.00       0.45
1                    14.0                 132.0  0.99400  3.30       0.49
2                    30.0                  97.0  0.99510  3.26       0.44
3                    47.0                 186.0  0.99560  3.19       0.40
4                    47.0                 186.0  0.99560  3.19       0.40
{\ldots}                   {\ldots}                   {\ldots}      {\ldots}   {\ldots}        {\ldots}
4893                 24.0                  92.0  0.99114  3.27       0.50
4894                 57.0                 168.0  0.99490  3.15       0.46
4895                 30.0                 111.0  0.99254  2.99       0.46
4896                 20.0                 110.0  0.98869  3.34       0.38
4897                 22.0                  98.0  0.98941  3.26       0.32

      alcohol  quality
0         8.8        6
1         9.5        6
2        10.1        6
3         9.9        6
4         9.9        6
{\ldots}       {\ldots}      {\ldots}
4893     11.2        6
4894      9.6        5
4895      9.4        6
4896     12.8        7
4897     11.8        6

[4898 rows x 12 columns]>
    \end{Verbatim}

    \begin{tcolorbox}[breakable, size=fbox, boxrule=1pt, pad at break*=1mm,colback=cellbackground, colframe=cellborder]
\prompt{In}{incolor}{3}{\boxspacing}
\begin{Verbatim}[commandchars=\\\{\}]
\PY{n+nb}{print}\PY{p}{(}\PY{n}{red\PYZus{}wine}\PY{o}{.}\PY{n}{info}\PY{p}{)}
\end{Verbatim}
\end{tcolorbox}

    \begin{Verbatim}[commandchars=\\\{\}]
<bound method DataFrame.info of       fixed acidity  volatile acidity  citric
acid  residual sugar  chlorides  \textbackslash{}
0               7.4             0.700         0.00             1.9      0.076
1               7.8             0.880         0.00             2.6      0.098
2               7.8             0.760         0.04             2.3      0.092
3              11.2             0.280         0.56             1.9      0.075
4               7.4             0.700         0.00             1.9      0.076
{\ldots}             {\ldots}               {\ldots}          {\ldots}             {\ldots}        {\ldots}
1594            6.2             0.600         0.08             2.0      0.090
1595            5.9             0.550         0.10             2.2      0.062
1596            6.3             0.510         0.13             2.3      0.076
1597            5.9             0.645         0.12             2.0      0.075
1598            6.0             0.310         0.47             3.6      0.067

      free sulfur dioxide  total sulfur dioxide  density    pH  sulphates  \textbackslash{}
0                    11.0                  34.0  0.99780  3.51       0.56
1                    25.0                  67.0  0.99680  3.20       0.68
2                    15.0                  54.0  0.99700  3.26       0.65
3                    17.0                  60.0  0.99800  3.16       0.58
4                    11.0                  34.0  0.99780  3.51       0.56
{\ldots}                   {\ldots}                   {\ldots}      {\ldots}   {\ldots}        {\ldots}
1594                 32.0                  44.0  0.99490  3.45       0.58
1595                 39.0                  51.0  0.99512  3.52       0.76
1596                 29.0                  40.0  0.99574  3.42       0.75
1597                 32.0                  44.0  0.99547  3.57       0.71
1598                 18.0                  42.0  0.99549  3.39       0.66

      alcohol  quality
0         9.4        5
1         9.8        5
2         9.8        5
3         9.8        6
4         9.4        5
{\ldots}       {\ldots}      {\ldots}
1594     10.5        5
1595     11.2        6
1596     11.0        6
1597     10.2        5
1598     11.0        6

[1599 rows x 12 columns]>
    \end{Verbatim}

    \begin{tcolorbox}[breakable, size=fbox, boxrule=1pt, pad at break*=1mm,colback=cellbackground, colframe=cellborder]
\prompt{In}{incolor}{4}{\boxspacing}
\begin{Verbatim}[commandchars=\\\{\}]
\PY{n}{red\PYZus{}wine}\PY{o}{.}\PY{n}{head}\PY{p}{(}\PY{p}{)}
\end{Verbatim}
\end{tcolorbox}

            \begin{tcolorbox}[breakable, size=fbox, boxrule=.5pt, pad at break*=1mm, opacityfill=0]
\prompt{Out}{outcolor}{4}{\boxspacing}
\begin{Verbatim}[commandchars=\\\{\}]
   fixed acidity  volatile acidity  citric acid  residual sugar  chlorides  \textbackslash{}
0            7.4              0.70         0.00             1.9      0.076
1            7.8              0.88         0.00             2.6      0.098
2            7.8              0.76         0.04             2.3      0.092
3           11.2              0.28         0.56             1.9      0.075
4            7.4              0.70         0.00             1.9      0.076

   free sulfur dioxide  total sulfur dioxide  density    pH  sulphates  \textbackslash{}
0                 11.0                  34.0   0.9978  3.51       0.56
1                 25.0                  67.0   0.9968  3.20       0.68
2                 15.0                  54.0   0.9970  3.26       0.65
3                 17.0                  60.0   0.9980  3.16       0.58
4                 11.0                  34.0   0.9978  3.51       0.56

   alcohol  quality
0      9.4        5
1      9.8        5
2      9.8        5
3      9.8        6
4      9.4        5
\end{Verbatim}
\end{tcolorbox}
        
    \begin{tcolorbox}[breakable, size=fbox, boxrule=1pt, pad at break*=1mm,colback=cellbackground, colframe=cellborder]
\prompt{In}{incolor}{5}{\boxspacing}
\begin{Verbatim}[commandchars=\\\{\}]
\PY{n}{white\PYZus{}wine}\PY{o}{.}\PY{n}{tail}\PY{p}{(}\PY{p}{)}
\end{Verbatim}
\end{tcolorbox}

            \begin{tcolorbox}[breakable, size=fbox, boxrule=.5pt, pad at break*=1mm, opacityfill=0]
\prompt{Out}{outcolor}{5}{\boxspacing}
\begin{Verbatim}[commandchars=\\\{\}]
      fixed acidity  volatile acidity  citric acid  residual sugar  chlorides  \textbackslash{}
4893            6.2              0.21         0.29             1.6      0.039
4894            6.6              0.32         0.36             8.0      0.047
4895            6.5              0.24         0.19             1.2      0.041
4896            5.5              0.29         0.30             1.1      0.022
4897            6.0              0.21         0.38             0.8      0.020

      free sulfur dioxide  total sulfur dioxide  density    pH  sulphates  \textbackslash{}
4893                 24.0                  92.0  0.99114  3.27       0.50
4894                 57.0                 168.0  0.99490  3.15       0.46
4895                 30.0                 111.0  0.99254  2.99       0.46
4896                 20.0                 110.0  0.98869  3.34       0.38
4897                 22.0                  98.0  0.98941  3.26       0.32

      alcohol  quality
4893     11.2        6
4894      9.6        5
4895      9.4        6
4896     12.8        7
4897     11.8        6
\end{Verbatim}
\end{tcolorbox}
        
    \begin{tcolorbox}[breakable, size=fbox, boxrule=1pt, pad at break*=1mm,colback=cellbackground, colframe=cellborder]
\prompt{In}{incolor}{6}{\boxspacing}
\begin{Verbatim}[commandchars=\\\{\}]
\PY{n}{red\PYZus{}wine}\PY{o}{.}\PY{n}{head}\PY{p}{(}\PY{l+m+mi}{5}\PY{p}{)}
\end{Verbatim}
\end{tcolorbox}

            \begin{tcolorbox}[breakable, size=fbox, boxrule=.5pt, pad at break*=1mm, opacityfill=0]
\prompt{Out}{outcolor}{6}{\boxspacing}
\begin{Verbatim}[commandchars=\\\{\}]
   fixed acidity  volatile acidity  citric acid  residual sugar  chlorides  \textbackslash{}
0            7.4              0.70         0.00             1.9      0.076
1            7.8              0.88         0.00             2.6      0.098
2            7.8              0.76         0.04             2.3      0.092
3           11.2              0.28         0.56             1.9      0.075
4            7.4              0.70         0.00             1.9      0.076

   free sulfur dioxide  total sulfur dioxide  density    pH  sulphates  \textbackslash{}
0                 11.0                  34.0   0.9978  3.51       0.56
1                 25.0                  67.0   0.9968  3.20       0.68
2                 15.0                  54.0   0.9970  3.26       0.65
3                 17.0                  60.0   0.9980  3.16       0.58
4                 11.0                  34.0   0.9978  3.51       0.56

   alcohol  quality
0      9.4        5
1      9.8        5
2      9.8        5
3      9.8        6
4      9.4        5
\end{Verbatim}
\end{tcolorbox}
        
    \begin{tcolorbox}[breakable, size=fbox, boxrule=1pt, pad at break*=1mm,colback=cellbackground, colframe=cellborder]
\prompt{In}{incolor}{7}{\boxspacing}
\begin{Verbatim}[commandchars=\\\{\}]
\PY{n}{white\PYZus{}wine}\PY{o}{.}\PY{n}{describe}\PY{p}{(}\PY{p}{)}
\end{Verbatim}
\end{tcolorbox}

            \begin{tcolorbox}[breakable, size=fbox, boxrule=.5pt, pad at break*=1mm, opacityfill=0]
\prompt{Out}{outcolor}{7}{\boxspacing}
\begin{Verbatim}[commandchars=\\\{\}]
       fixed acidity  volatile acidity  citric acid  residual sugar  \textbackslash{}
count    4898.000000       4898.000000  4898.000000     4898.000000
mean        6.854788          0.278241     0.334192        6.391415
std         0.843868          0.100795     0.121020        5.072058
min         3.800000          0.080000     0.000000        0.600000
25\%         6.300000          0.210000     0.270000        1.700000
50\%         6.800000          0.260000     0.320000        5.200000
75\%         7.300000          0.320000     0.390000        9.900000
max        14.200000          1.100000     1.660000       65.800000

         chlorides  free sulfur dioxide  total sulfur dioxide      density  \textbackslash{}
count  4898.000000          4898.000000           4898.000000  4898.000000
mean      0.045772            35.308085            138.360657     0.994027
std       0.021848            17.007137             42.498065     0.002991
min       0.009000             2.000000              9.000000     0.987110
25\%       0.036000            23.000000            108.000000     0.991723
50\%       0.043000            34.000000            134.000000     0.993740
75\%       0.050000            46.000000            167.000000     0.996100
max       0.346000           289.000000            440.000000     1.038980

                pH    sulphates      alcohol      quality
count  4898.000000  4898.000000  4898.000000  4898.000000
mean      3.188267     0.489847    10.514267     5.877909
std       0.151001     0.114126     1.230621     0.885639
min       2.720000     0.220000     8.000000     3.000000
25\%       3.090000     0.410000     9.500000     5.000000
50\%       3.180000     0.470000    10.400000     6.000000
75\%       3.280000     0.550000    11.400000     6.000000
max       3.820000     1.080000    14.200000     9.000000
\end{Verbatim}
\end{tcolorbox}
        
    \begin{tcolorbox}[breakable, size=fbox, boxrule=1pt, pad at break*=1mm,colback=cellbackground, colframe=cellborder]
\prompt{In}{incolor}{8}{\boxspacing}
\begin{Verbatim}[commandchars=\\\{\}]
\PY{n}{pd}\PY{o}{.}\PY{n}{isnull}\PY{p}{(}\PY{n}{red\PYZus{}wine}\PY{p}{)}
\end{Verbatim}
\end{tcolorbox}

            \begin{tcolorbox}[breakable, size=fbox, boxrule=.5pt, pad at break*=1mm, opacityfill=0]
\prompt{Out}{outcolor}{8}{\boxspacing}
\begin{Verbatim}[commandchars=\\\{\}]
      fixed acidity  volatile acidity  citric acid  residual sugar  chlorides  \textbackslash{}
0             False             False        False           False      False
1             False             False        False           False      False
2             False             False        False           False      False
3             False             False        False           False      False
4             False             False        False           False      False
{\ldots}             {\ldots}               {\ldots}          {\ldots}             {\ldots}        {\ldots}
1594          False             False        False           False      False
1595          False             False        False           False      False
1596          False             False        False           False      False
1597          False             False        False           False      False
1598          False             False        False           False      False

      free sulfur dioxide  total sulfur dioxide  density     pH  sulphates  \textbackslash{}
0                   False                 False    False  False      False
1                   False                 False    False  False      False
2                   False                 False    False  False      False
3                   False                 False    False  False      False
4                   False                 False    False  False      False
{\ldots}                   {\ldots}                   {\ldots}      {\ldots}    {\ldots}        {\ldots}
1594                False                 False    False  False      False
1595                False                 False    False  False      False
1596                False                 False    False  False      False
1597                False                 False    False  False      False
1598                False                 False    False  False      False

      alcohol  quality
0       False    False
1       False    False
2       False    False
3       False    False
4       False    False
{\ldots}       {\ldots}      {\ldots}
1594    False    False
1595    False    False
1596    False    False
1597    False    False
1598    False    False

[1599 rows x 12 columns]
\end{Verbatim}
\end{tcolorbox}
        
    \begin{tcolorbox}[breakable, size=fbox, boxrule=1pt, pad at break*=1mm,colback=cellbackground, colframe=cellborder]
\prompt{In}{incolor}{9}{\boxspacing}
\begin{Verbatim}[commandchars=\\\{\}]
\PY{n}{pd}\PY{o}{.}\PY{n}{isnull}\PY{p}{(}\PY{n}{white\PYZus{}wine}\PY{p}{)}
\end{Verbatim}
\end{tcolorbox}

            \begin{tcolorbox}[breakable, size=fbox, boxrule=.5pt, pad at break*=1mm, opacityfill=0]
\prompt{Out}{outcolor}{9}{\boxspacing}
\begin{Verbatim}[commandchars=\\\{\}]
      fixed acidity  volatile acidity  citric acid  residual sugar  chlorides  \textbackslash{}
0             False             False        False           False      False
1             False             False        False           False      False
2             False             False        False           False      False
3             False             False        False           False      False
4             False             False        False           False      False
{\ldots}             {\ldots}               {\ldots}          {\ldots}             {\ldots}        {\ldots}
4893          False             False        False           False      False
4894          False             False        False           False      False
4895          False             False        False           False      False
4896          False             False        False           False      False
4897          False             False        False           False      False

      free sulfur dioxide  total sulfur dioxide  density     pH  sulphates  \textbackslash{}
0                   False                 False    False  False      False
1                   False                 False    False  False      False
2                   False                 False    False  False      False
3                   False                 False    False  False      False
4                   False                 False    False  False      False
{\ldots}                   {\ldots}                   {\ldots}      {\ldots}    {\ldots}        {\ldots}
4893                False                 False    False  False      False
4894                False                 False    False  False      False
4895                False                 False    False  False      False
4896                False                 False    False  False      False
4897                False                 False    False  False      False

      alcohol  quality
0       False    False
1       False    False
2       False    False
3       False    False
4       False    False
{\ldots}       {\ldots}      {\ldots}
4893    False    False
4894    False    False
4895    False    False
4896    False    False
4897    False    False

[4898 rows x 12 columns]
\end{Verbatim}
\end{tcolorbox}
        
    This subsection of code allows to create a comparison between histograms
of both red and white wines, taking into account the \% of alcohol in
the wine.

    \begin{tcolorbox}[breakable, size=fbox, boxrule=1pt, pad at break*=1mm,colback=cellbackground, colframe=cellborder]
\prompt{In}{incolor}{10}{\boxspacing}
\begin{Verbatim}[commandchars=\\\{\}]
\PY{k+kn}{import} \PY{n+nn}{matplotlib}\PY{n+nn}{.}\PY{n+nn}{pyplot} \PY{k}{as} \PY{n+nn}{plt}
\PY{n}{fig}\PY{p}{,}\PY{n}{ax}\PY{o}{=}\PY{n}{plt}\PY{o}{.}\PY{n}{subplots}\PY{p}{(}\PY{l+m+mi}{1}\PY{p}{,}\PY{l+m+mi}{2}\PY{p}{)}
\PY{n}{ax}\PY{p}{[}\PY{l+m+mi}{0}\PY{p}{]}\PY{o}{.}\PY{n}{hist}\PY{p}{(}\PY{n}{red\PYZus{}wine}\PY{o}{.}\PY{n}{alcohol}\PY{p}{,}\PY{l+m+mi}{10}\PY{p}{,}\PY{n}{facecolor}\PY{o}{=}\PY{l+s+s1}{\PYZsq{}}\PY{l+s+s1}{red}\PY{l+s+s1}{\PYZsq{}}\PY{p}{,}\PY{n}{alpha}\PY{o}{=}\PY{l+m+mf}{0.5}\PY{p}{,}\PY{n}{label}\PY{o}{=}\PY{l+s+s2}{\PYZdq{}}\PY{l+s+s2}{Red Wine}\PY{l+s+s2}{\PYZdq{}}\PY{p}{)}
\PY{n}{ax}\PY{p}{[}\PY{l+m+mi}{1}\PY{p}{]}\PY{o}{.}\PY{n}{hist}\PY{p}{(}\PY{n}{white\PYZus{}wine}\PY{o}{.}\PY{n}{alcohol}\PY{p}{,}\PY{l+m+mi}{10}\PY{p}{,}\PY{n}{facecolor}\PY{o}{=}\PY{l+s+s1}{\PYZsq{}}\PY{l+s+s1}{white}\PY{l+s+s1}{\PYZsq{}}\PY{p}{,}\PY{n}{ec}\PY{o}{=}\PY{l+s+s1}{\PYZsq{}}\PY{l+s+s1}{black}\PY{l+s+s1}{\PYZsq{}}\PY{p}{,}\PY{n}{lw}\PY{o}{=}\PY{l+m+mf}{0.5}\PY{p}{,}\PY{n}{alpha}\PY{o}{=}\PY{l+m+mf}{0.5}\PY{p}{,}\PY{n}{label}\PY{o}{=}\PY{l+s+s2}{\PYZdq{}}\PY{l+s+s2}{White Wine}\PY{l+s+s2}{\PYZdq{}}\PY{p}{)}
\PY{n}{fig}\PY{o}{.}\PY{n}{subplots\PYZus{}adjust}\PY{p}{(}\PY{n}{left}\PY{o}{=}\PY{l+m+mi}{0}\PY{p}{,}\PY{n}{right}\PY{o}{=}\PY{l+m+mi}{1}\PY{p}{,}\PY{n}{bottom}\PY{o}{=}\PY{l+m+mi}{0}\PY{p}{,}\PY{n}{top}\PY{o}{=}\PY{l+m+mf}{0.5}\PY{p}{,}\PY{n}{hspace}\PY{o}{=}\PY{l+m+mf}{0.05}\PY{p}{,}\PY{n}{wspace}\PY{o}{=}\PY{l+m+mi}{1}\PY{p}{)}
\PY{n}{ax}\PY{p}{[}\PY{l+m+mi}{0}\PY{p}{]}\PY{o}{.}\PY{n}{set\PYZus{}ylim}\PY{p}{(}\PY{p}{[}\PY{l+m+mi}{0}\PY{p}{,}\PY{l+m+mi}{1000}\PY{p}{]}\PY{p}{)}
\PY{n}{ax}\PY{p}{[}\PY{l+m+mi}{0}\PY{p}{]}\PY{o}{.}\PY{n}{set\PYZus{}xlabel}\PY{p}{(}\PY{l+s+s2}{\PYZdq{}}\PY{l+s+s2}{Alcohol in }\PY{l+s+s2}{\PYZpc{}}\PY{l+s+s2}{ Vol}\PY{l+s+s2}{\PYZdq{}}\PY{p}{)}
\PY{n}{ax}\PY{p}{[}\PY{l+m+mi}{0}\PY{p}{]}\PY{o}{.}\PY{n}{set\PYZus{}ylabel}\PY{p}{(}\PY{l+s+s2}{\PYZdq{}}\PY{l+s+s2}{Frequency}\PY{l+s+s2}{\PYZdq{}}\PY{p}{)}
\PY{n}{ax}\PY{p}{[}\PY{l+m+mi}{1}\PY{p}{]}\PY{o}{.}\PY{n}{set\PYZus{}xlabel}\PY{p}{(}\PY{l+s+s2}{\PYZdq{}}\PY{l+s+s2}{Alcohol in }\PY{l+s+s2}{\PYZpc{}}\PY{l+s+s2}{ Vol}\PY{l+s+s2}{\PYZdq{}}\PY{p}{)}
\PY{n}{ax}\PY{p}{[}\PY{l+m+mi}{1}\PY{p}{]}\PY{o}{.}\PY{n}{set\PYZus{}ylabel}\PY{p}{(}\PY{l+s+s2}{\PYZdq{}}\PY{l+s+s2}{Frequency}\PY{l+s+s2}{\PYZdq{}}\PY{p}{)}
\PY{n}{fig}\PY{o}{.}\PY{n}{suptitle}\PY{p}{(}\PY{l+s+s2}{\PYZdq{}}\PY{l+s+s2}{Distribution of Alcohol in }\PY{l+s+s2}{\PYZpc{}}\PY{l+s+s2}{ Vol}\PY{l+s+s2}{\PYZdq{}}\PY{p}{)}

\PY{n}{plt}\PY{o}{.}\PY{n}{show}\PY{p}{(}\PY{p}{)}
\end{Verbatim}
\end{tcolorbox}

    
    \begin{verbatim}
<Figure size 640x480 with 2 Axes>
    \end{verbatim}

    
    \begin{tcolorbox}[breakable, size=fbox, boxrule=1pt, pad at break*=1mm,colback=cellbackground, colframe=cellborder]
\prompt{In}{incolor}{11}{\boxspacing}
\begin{Verbatim}[commandchars=\\\{\}]
\PY{k+kn}{import} \PY{n+nn}{numpy} \PY{k}{as} \PY{n+nn}{np}
\PY{n+nb}{print}\PY{p}{(}\PY{n}{np}\PY{o}{.}\PY{n}{histogram}\PY{p}{(}\PY{n}{red\PYZus{}wine}\PY{o}{.}\PY{n}{alcohol}\PY{p}{,}\PY{n}{bins}\PY{o}{=}\PY{p}{[}\PY{l+m+mi}{7}\PY{p}{,}\PY{l+m+mi}{8}\PY{p}{,}\PY{l+m+mi}{9}\PY{p}{,}\PY{l+m+mi}{10}\PY{p}{,}\PY{l+m+mi}{11}\PY{p}{,}\PY{l+m+mi}{12}\PY{p}{,}\PY{l+m+mi}{13}\PY{p}{,}\PY{l+m+mi}{14}\PY{p}{,}\PY{l+m+mi}{15}\PY{p}{]}\PY{p}{)}\PY{p}{)}
\PY{n+nb}{print}\PY{p}{(}\PY{n}{np}\PY{o}{.}\PY{n}{histogram}\PY{p}{(}\PY{n}{white\PYZus{}wine}\PY{o}{.}\PY{n}{alcohol}\PY{p}{,}\PY{n}{bins}\PY{o}{=}\PY{p}{[}\PY{l+m+mi}{7}\PY{p}{,}\PY{l+m+mi}{8}\PY{p}{,}\PY{l+m+mi}{9}\PY{p}{,}\PY{l+m+mi}{10}\PY{p}{,}\PY{l+m+mi}{11}\PY{p}{,}\PY{l+m+mi}{12}\PY{p}{,}\PY{l+m+mi}{13}\PY{p}{,}\PY{l+m+mi}{14}\PY{p}{,}\PY{l+m+mi}{15}\PY{p}{]}\PY{p}{)}\PY{p}{)}
\end{Verbatim}
\end{tcolorbox}

    \begin{Verbatim}[commandchars=\\\{\}]
(array([  0,   7, 673, 452, 305, 133,  21,   8], dtype=int64), array([ 7,  8,
9, 10, 11, 12, 13, 14, 15]))
(array([   0,  317, 1606, 1256,  906,  675,  131,    7], dtype=int64), array([
7,  8,  9, 10, 11, 12, 13, 14, 15]))
    \end{Verbatim}

    Another comparison, this time comparing Quality of the wines and the
amount of sulphates in them

    \begin{tcolorbox}[breakable, size=fbox, boxrule=1pt, pad at break*=1mm,colback=cellbackground, colframe=cellborder]
\prompt{In}{incolor}{12}{\boxspacing}
\begin{Verbatim}[commandchars=\\\{\}]
\PY{k+kn}{import} \PY{n+nn}{matplotlib}\PY{n+nn}{.}\PY{n+nn}{pyplot} \PY{k}{as} \PY{n+nn}{plt}
\PY{n}{fig}\PY{p}{,}\PY{n}{ax}\PY{o}{=}\PY{n}{plt}\PY{o}{.}\PY{n}{subplots}\PY{p}{(}\PY{l+m+mi}{1}\PY{p}{,}\PY{l+m+mi}{2}\PY{p}{,} \PY{n}{figsize}\PY{o}{=}\PY{p}{(}\PY{l+m+mi}{8}\PY{p}{,}\PY{l+m+mi}{4}\PY{p}{)}\PY{p}{)}

\PY{n}{ax}\PY{p}{[}\PY{l+m+mi}{0}\PY{p}{]}\PY{o}{.}\PY{n}{scatter}\PY{p}{(}\PY{n}{red\PYZus{}wine}\PY{p}{[}\PY{l+s+s1}{\PYZsq{}}\PY{l+s+s1}{quality}\PY{l+s+s1}{\PYZsq{}}\PY{p}{]}\PY{p}{,}\PY{n}{red\PYZus{}wine}\PY{p}{[}\PY{l+s+s1}{\PYZsq{}}\PY{l+s+s1}{sulphates}\PY{l+s+s1}{\PYZsq{}}\PY{p}{]}\PY{p}{,}\PY{n}{color}\PY{o}{=}\PY{l+s+s1}{\PYZsq{}}\PY{l+s+s1}{red}\PY{l+s+s1}{\PYZsq{}}\PY{p}{)}
\PY{n}{ax}\PY{p}{[}\PY{l+m+mi}{1}\PY{p}{]}\PY{o}{.}\PY{n}{scatter}\PY{p}{(}\PY{n}{white\PYZus{}wine}\PY{p}{[}\PY{l+s+s1}{\PYZsq{}}\PY{l+s+s1}{quality}\PY{l+s+s1}{\PYZsq{}}\PY{p}{]}\PY{p}{,}\PY{n}{white\PYZus{}wine}\PY{p}{[}\PY{l+s+s1}{\PYZsq{}}\PY{l+s+s1}{sulphates}\PY{l+s+s1}{\PYZsq{}}\PY{p}{]}\PY{p}{,}\PY{n}{color}\PY{o}{=}\PY{l+s+s1}{\PYZsq{}}\PY{l+s+s1}{white}\PY{l+s+s1}{\PYZsq{}}\PY{p}{,}\PY{n}{edgecolors}\PY{o}{=}\PY{l+s+s1}{\PYZsq{}}\PY{l+s+s1}{black}\PY{l+s+s1}{\PYZsq{}}\PY{p}{,}\PY{n}{lw}\PY{o}{=}\PY{l+m+mf}{0.5}\PY{p}{)}

\PY{n}{ax}\PY{p}{[}\PY{l+m+mi}{0}\PY{p}{]}\PY{o}{.}\PY{n}{set\PYZus{}title}\PY{p}{(}\PY{l+s+s2}{\PYZdq{}}\PY{l+s+s2}{Red Wine}\PY{l+s+s2}{\PYZdq{}}\PY{p}{)}
\PY{n}{ax}\PY{p}{[}\PY{l+m+mi}{1}\PY{p}{]}\PY{o}{.}\PY{n}{set\PYZus{}title}\PY{p}{(}\PY{l+s+s2}{\PYZdq{}}\PY{l+s+s2}{White Wine}\PY{l+s+s2}{\PYZdq{}}\PY{p}{)}
\PY{n}{ax}\PY{p}{[}\PY{l+m+mi}{0}\PY{p}{]}\PY{o}{.}\PY{n}{set\PYZus{}xlabel}\PY{p}{(}\PY{l+s+s2}{\PYZdq{}}\PY{l+s+s2}{Quality}\PY{l+s+s2}{\PYZdq{}}\PY{p}{)}
\PY{n}{ax}\PY{p}{[}\PY{l+m+mi}{1}\PY{p}{]}\PY{o}{.}\PY{n}{set\PYZus{}xlabel}\PY{p}{(}\PY{l+s+s2}{\PYZdq{}}\PY{l+s+s2}{Quality}\PY{l+s+s2}{\PYZdq{}}\PY{p}{)}
\PY{n}{ax}\PY{p}{[}\PY{l+m+mi}{0}\PY{p}{]}\PY{o}{.}\PY{n}{set\PYZus{}ylabel}\PY{p}{(}\PY{l+s+s2}{\PYZdq{}}\PY{l+s+s2}{Sulphates}\PY{l+s+s2}{\PYZdq{}}\PY{p}{)}
\PY{n}{ax}\PY{p}{[}\PY{l+m+mi}{1}\PY{p}{]}\PY{o}{.}\PY{n}{set\PYZus{}ylabel}\PY{p}{(}\PY{l+s+s2}{\PYZdq{}}\PY{l+s+s2}{Sulphates}\PY{l+s+s2}{\PYZdq{}}\PY{p}{)}
\PY{n}{ax}\PY{p}{[}\PY{l+m+mi}{0}\PY{p}{]}\PY{o}{.}\PY{n}{set\PYZus{}xlim}\PY{p}{(}\PY{p}{[}\PY{l+m+mi}{0}\PY{p}{,}\PY{l+m+mi}{10}\PY{p}{]}\PY{p}{)}
\PY{n}{ax}\PY{p}{[}\PY{l+m+mi}{1}\PY{p}{]}\PY{o}{.}\PY{n}{set\PYZus{}xlim}\PY{p}{(}\PY{p}{[}\PY{l+m+mi}{0}\PY{p}{,}\PY{l+m+mi}{10}\PY{p}{]}\PY{p}{)}
\PY{n}{ax}\PY{p}{[}\PY{l+m+mi}{0}\PY{p}{]}\PY{o}{.}\PY{n}{set\PYZus{}ylim}\PY{p}{(}\PY{p}{[}\PY{l+m+mi}{0}\PY{p}{,}\PY{l+m+mf}{2.5}\PY{p}{]}\PY{p}{)}
\PY{n}{ax}\PY{p}{[}\PY{l+m+mi}{1}\PY{p}{]}\PY{o}{.}\PY{n}{set\PYZus{}ylim}\PY{p}{(}\PY{p}{[}\PY{l+m+mi}{0}\PY{p}{,}\PY{l+m+mf}{2.5}\PY{p}{]}\PY{p}{)}
\PY{n}{fig}\PY{o}{.}\PY{n}{subplots\PYZus{}adjust}\PY{p}{(}\PY{n}{wspace}\PY{o}{=}\PY{l+m+mf}{0.5}\PY{p}{)}
\PY{n}{fig}\PY{o}{.}\PY{n}{suptitle}\PY{p}{(}\PY{l+s+s2}{\PYZdq{}}\PY{l+s+s2}{Wine Quality by Amount of Sulphates}\PY{l+s+s2}{\PYZdq{}}\PY{p}{)}

\PY{n}{plt}\PY{o}{.}\PY{n}{show}\PY{p}{(}\PY{p}{)}
\end{Verbatim}
\end{tcolorbox}

    \begin{center}
    \adjustimage{max size={0.9\linewidth}{0.9\paperheight}}{output_14_0.png}
    \end{center}
    { \hspace*{\fill} \\}
    
    And a third comparison, this time using quality and the volatile acidity

    \begin{tcolorbox}[breakable, size=fbox, boxrule=1pt, pad at break*=1mm,colback=cellbackground, colframe=cellborder]
\prompt{In}{incolor}{13}{\boxspacing}
\begin{Verbatim}[commandchars=\\\{\}]
\PY{k+kn}{import} \PY{n+nn}{matplotlib}\PY{n+nn}{.}\PY{n+nn}{pyplot} \PY{k}{as} \PY{n+nn}{plt}
\PY{k+kn}{import} \PY{n+nn}{numpy} \PY{k}{as} \PY{n+nn}{np}
\PY{n}{np}\PY{o}{.}\PY{n}{random}\PY{o}{.}\PY{n}{seed}\PY{p}{(}\PY{l+m+mi}{570}\PY{p}{)}
\PY{n}{redlabels}\PY{o}{=}\PY{n}{np}\PY{o}{.}\PY{n}{unique}\PY{p}{(}\PY{n}{red\PYZus{}wine}\PY{p}{[}\PY{l+s+s1}{\PYZsq{}}\PY{l+s+s1}{quality}\PY{l+s+s1}{\PYZsq{}}\PY{p}{]}\PY{p}{)}
\PY{n}{whitelabels}\PY{o}{=}\PY{n}{np}\PY{o}{.}\PY{n}{unique}\PY{p}{(}\PY{n}{white\PYZus{}wine}\PY{p}{[}\PY{l+s+s1}{\PYZsq{}}\PY{l+s+s1}{quality}\PY{l+s+s1}{\PYZsq{}}\PY{p}{]}\PY{p}{)}

\PY{n}{fig}\PY{p}{,}\PY{n}{ax}\PY{o}{=}\PY{n}{plt}\PY{o}{.}\PY{n}{subplots}\PY{p}{(}\PY{l+m+mi}{1}\PY{p}{,}\PY{l+m+mi}{2}\PY{p}{,}\PY{n}{figsize}\PY{o}{=}\PY{p}{(}\PY{l+m+mi}{8}\PY{p}{,}\PY{l+m+mi}{4}\PY{p}{)}\PY{p}{)}
\PY{n}{redcolors}\PY{o}{=}\PY{n}{np}\PY{o}{.}\PY{n}{random}\PY{o}{.}\PY{n}{rand}\PY{p}{(}\PY{l+m+mi}{6}\PY{p}{,}\PY{l+m+mi}{4}\PY{p}{)}
\PY{n}{whitecolors}\PY{o}{=}\PY{n}{np}\PY{o}{.}\PY{n}{append}\PY{p}{(}\PY{n}{redcolors}\PY{p}{,}\PY{n}{np}\PY{o}{.}\PY{n}{random}\PY{o}{.}\PY{n}{rand}\PY{p}{(}\PY{l+m+mi}{1}\PY{p}{,}\PY{l+m+mi}{4}\PY{p}{)}\PY{p}{,}\PY{n}{axis}\PY{o}{=}\PY{l+m+mi}{0}\PY{p}{)}

\PY{k}{for} \PY{n}{i} \PY{o+ow}{in} \PY{n+nb}{range}\PY{p}{(}\PY{n+nb}{len}\PY{p}{(}\PY{n}{redcolors}\PY{p}{)}\PY{p}{)}\PY{p}{:}
    \PY{n}{redy}\PY{o}{=}\PY{n}{red\PYZus{}wine}\PY{p}{[}\PY{l+s+s1}{\PYZsq{}}\PY{l+s+s1}{alcohol}\PY{l+s+s1}{\PYZsq{}}\PY{p}{]}\PY{p}{[}\PY{n}{red\PYZus{}wine}\PY{o}{.}\PY{n}{quality}\PY{o}{==}\PY{n}{redlabels}\PY{p}{[}\PY{n}{i}\PY{p}{]}\PY{p}{]}
    \PY{n}{redx}\PY{o}{=}\PY{n}{red\PYZus{}wine}\PY{p}{[}\PY{l+s+s1}{\PYZsq{}}\PY{l+s+s1}{volatile acidity}\PY{l+s+s1}{\PYZsq{}}\PY{p}{]}\PY{p}{[}\PY{n}{red\PYZus{}wine}\PY{o}{.}\PY{n}{quality}\PY{o}{==} \PY{n}{redlabels}\PY{p}{[}\PY{n}{i}\PY{p}{]}\PY{p}{]}
    \PY{n}{ax}\PY{p}{[}\PY{l+m+mi}{0}\PY{p}{]}\PY{o}{.}\PY{n}{scatter}\PY{p}{(}\PY{n}{redx}\PY{p}{,}\PY{n}{redy}\PY{p}{,}\PY{n}{c}\PY{o}{=}\PY{n}{redcolors}\PY{p}{[}\PY{n}{i}\PY{p}{]}\PY{p}{)}
\PY{k}{for} \PY{n}{i} \PY{o+ow}{in} \PY{n+nb}{range}\PY{p}{(}\PY{n+nb}{len}\PY{p}{(}\PY{n}{whitecolors}\PY{p}{)}\PY{p}{)}\PY{p}{:}
    \PY{n}{whitey}\PY{o}{=}\PY{n}{white\PYZus{}wine}\PY{p}{[}\PY{l+s+s1}{\PYZsq{}}\PY{l+s+s1}{alcohol}\PY{l+s+s1}{\PYZsq{}}\PY{p}{]}\PY{p}{[}\PY{n}{white\PYZus{}wine}\PY{o}{.}\PY{n}{quality}\PY{o}{==}\PY{n}{whitelabels}\PY{p}{[}\PY{n}{i}\PY{p}{]}\PY{p}{]}
    \PY{n}{whitex}\PY{o}{=}\PY{n}{white\PYZus{}wine}\PY{p}{[}\PY{l+s+s1}{\PYZsq{}}\PY{l+s+s1}{volatile acidity}\PY{l+s+s1}{\PYZsq{}}\PY{p}{]}\PY{p}{[}\PY{n}{white\PYZus{}wine}\PY{o}{.}\PY{n}{quality}\PY{o}{==} \PY{n}{whitelabels}\PY{p}{[}\PY{n}{i}\PY{p}{]}\PY{p}{]}
    \PY{n}{ax}\PY{p}{[}\PY{l+m+mi}{1}\PY{p}{]}\PY{o}{.}\PY{n}{scatter}\PY{p}{(}\PY{n}{whitex}\PY{p}{,}\PY{n}{whitey}\PY{p}{,}\PY{n}{c}\PY{o}{=}\PY{n}{whitecolors}\PY{p}{[}\PY{n}{i}\PY{p}{]}\PY{p}{)}

\PY{n}{ax}\PY{p}{[}\PY{l+m+mi}{0}\PY{p}{]}\PY{o}{.}\PY{n}{set\PYZus{}title}\PY{p}{(}\PY{l+s+s2}{\PYZdq{}}\PY{l+s+s2}{Red Wine}\PY{l+s+s2}{\PYZdq{}}\PY{p}{)}
\PY{n}{ax}\PY{p}{[}\PY{l+m+mi}{1}\PY{p}{]}\PY{o}{.}\PY{n}{set\PYZus{}title}\PY{p}{(}\PY{l+s+s2}{\PYZdq{}}\PY{l+s+s2}{White Wine}\PY{l+s+s2}{\PYZdq{}}\PY{p}{)}
\PY{n}{ax}\PY{p}{[}\PY{l+m+mi}{0}\PY{p}{]}\PY{o}{.}\PY{n}{set\PYZus{}xlim}\PY{p}{(}\PY{p}{[}\PY{l+m+mi}{0}\PY{p}{,}\PY{l+m+mf}{1.7}\PY{p}{]}\PY{p}{)}
\PY{n}{ax}\PY{p}{[}\PY{l+m+mi}{1}\PY{p}{]}\PY{o}{.}\PY{n}{set\PYZus{}xlim}\PY{p}{(}\PY{p}{[}\PY{l+m+mi}{0}\PY{p}{,}\PY{l+m+mf}{1.7}\PY{p}{]}\PY{p}{)}
\PY{n}{ax}\PY{p}{[}\PY{l+m+mi}{0}\PY{p}{]}\PY{o}{.}\PY{n}{set\PYZus{}ylim}\PY{p}{(}\PY{p}{[}\PY{l+m+mi}{5}\PY{p}{,}\PY{l+m+mf}{15.5}\PY{p}{]}\PY{p}{)}
\PY{n}{ax}\PY{p}{[}\PY{l+m+mi}{1}\PY{p}{]}\PY{o}{.}\PY{n}{set\PYZus{}ylim}\PY{p}{(}\PY{p}{[}\PY{l+m+mi}{5}\PY{p}{,}\PY{l+m+mf}{15.5}\PY{p}{]}\PY{p}{)}
\PY{n}{ax}\PY{p}{[}\PY{l+m+mi}{0}\PY{p}{]}\PY{o}{.}\PY{n}{set\PYZus{}xlabel}\PY{p}{(}\PY{l+s+s2}{\PYZdq{}}\PY{l+s+s2}{Volatile Acidity}\PY{l+s+s2}{\PYZdq{}}\PY{p}{)}
\PY{n}{ax}\PY{p}{[}\PY{l+m+mi}{1}\PY{p}{]}\PY{o}{.}\PY{n}{set\PYZus{}xlabel}\PY{p}{(}\PY{l+s+s2}{\PYZdq{}}\PY{l+s+s2}{Volatile Acidity}\PY{l+s+s2}{\PYZdq{}}\PY{p}{)}
\PY{n}{ax}\PY{p}{[}\PY{l+m+mi}{0}\PY{p}{]}\PY{o}{.}\PY{n}{set\PYZus{}ylabel}\PY{p}{(}\PY{l+s+s2}{\PYZdq{}}\PY{l+s+s2}{Alcohol}\PY{l+s+s2}{\PYZdq{}}\PY{p}{)}
\PY{n}{ax}\PY{p}{[}\PY{l+m+mi}{1}\PY{p}{]}\PY{o}{.}\PY{n}{set\PYZus{}ylabel}\PY{p}{(}\PY{l+s+s2}{\PYZdq{}}\PY{l+s+s2}{Alcohol}\PY{l+s+s2}{\PYZdq{}}\PY{p}{)}

\PY{n}{ax}\PY{p}{[}\PY{l+m+mi}{1}\PY{p}{]}\PY{o}{.}\PY{n}{legend}\PY{p}{(}\PY{n}{whitelabels}\PY{p}{,}\PY{n}{loc}\PY{o}{=}\PY{l+s+s1}{\PYZsq{}}\PY{l+s+s1}{best}\PY{l+s+s1}{\PYZsq{}}\PY{p}{,}\PY{n}{bbox\PYZus{}to\PYZus{}anchor}\PY{o}{=}\PY{p}{(}\PY{l+m+mf}{1.3}\PY{p}{,}\PY{l+m+mi}{1}\PY{p}{)}\PY{p}{)}
\PY{n}{fig}\PY{o}{.}\PY{n}{subplots\PYZus{}adjust}\PY{p}{(}\PY{n}{top}\PY{o}{=}\PY{l+m+mf}{0.85}\PY{p}{,}\PY{n}{wspace}\PY{o}{=}\PY{l+m+mf}{0.7}\PY{p}{)}
\PY{n}{plt}\PY{o}{.}\PY{n}{show}\PY{p}{(}\PY{p}{)}
\end{Verbatim}
\end{tcolorbox}

    \begin{Verbatim}[commandchars=\\\{\}]
'c' argument looks like a single numeric RGB or RGBA sequence, which should be
avoided as value-mapping will have precedence in case its length matches with
'x' \& 'y'.  Please use a 2-D array with a single row if you really want to
specify the same RGB or RGBA value for all points.
'c' argument looks like a single numeric RGB or RGBA sequence, which should be
avoided as value-mapping will have precedence in case its length matches with
'x' \& 'y'.  Please use a 2-D array with a single row if you really want to
specify the same RGB or RGBA value for all points.
'c' argument looks like a single numeric RGB or RGBA sequence, which should be
avoided as value-mapping will have precedence in case its length matches with
'x' \& 'y'.  Please use a 2-D array with a single row if you really want to
specify the same RGB or RGBA value for all points.
'c' argument looks like a single numeric RGB or RGBA sequence, which should be
avoided as value-mapping will have precedence in case its length matches with
'x' \& 'y'.  Please use a 2-D array with a single row if you really want to
specify the same RGB or RGBA value for all points.
'c' argument looks like a single numeric RGB or RGBA sequence, which should be
avoided as value-mapping will have precedence in case its length matches with
'x' \& 'y'.  Please use a 2-D array with a single row if you really want to
specify the same RGB or RGBA value for all points.
'c' argument looks like a single numeric RGB or RGBA sequence, which should be
avoided as value-mapping will have precedence in case its length matches with
'x' \& 'y'.  Please use a 2-D array with a single row if you really want to
specify the same RGB or RGBA value for all points.
'c' argument looks like a single numeric RGB or RGBA sequence, which should be
avoided as value-mapping will have precedence in case its length matches with
'x' \& 'y'.  Please use a 2-D array with a single row if you really want to
specify the same RGB or RGBA value for all points.
'c' argument looks like a single numeric RGB or RGBA sequence, which should be
avoided as value-mapping will have precedence in case its length matches with
'x' \& 'y'.  Please use a 2-D array with a single row if you really want to
specify the same RGB or RGBA value for all points.
'c' argument looks like a single numeric RGB or RGBA sequence, which should be
avoided as value-mapping will have precedence in case its length matches with
'x' \& 'y'.  Please use a 2-D array with a single row if you really want to
specify the same RGB or RGBA value for all points.
'c' argument looks like a single numeric RGB or RGBA sequence, which should be
avoided as value-mapping will have precedence in case its length matches with
'x' \& 'y'.  Please use a 2-D array with a single row if you really want to
specify the same RGB or RGBA value for all points.
'c' argument looks like a single numeric RGB or RGBA sequence, which should be
avoided as value-mapping will have precedence in case its length matches with
'x' \& 'y'.  Please use a 2-D array with a single row if you really want to
specify the same RGB or RGBA value for all points.
'c' argument looks like a single numeric RGB or RGBA sequence, which should be
avoided as value-mapping will have precedence in case its length matches with
'x' \& 'y'.  Please use a 2-D array with a single row if you really want to
specify the same RGB or RGBA value for all points.
'c' argument looks like a single numeric RGB or RGBA sequence, which should be
avoided as value-mapping will have precedence in case its length matches with
'x' \& 'y'.  Please use a 2-D array with a single row if you really want to
specify the same RGB or RGBA value for all points.
    \end{Verbatim}

    \begin{center}
    \adjustimage{max size={0.9\linewidth}{0.9\paperheight}}{output_16_1.png}
    \end{center}
    { \hspace*{\fill} \\}
    
    In the next section of code, we build a heatmap using the correlation of
the atribues of the wines, determining which attributes are closely
related and which aren't.

    \begin{tcolorbox}[breakable, size=fbox, boxrule=1pt, pad at break*=1mm,colback=cellbackground, colframe=cellborder]
\prompt{In}{incolor}{14}{\boxspacing}
\begin{Verbatim}[commandchars=\\\{\}]
\PY{n}{red\PYZus{}wine}\PY{p}{[}\PY{l+s+s1}{\PYZsq{}}\PY{l+s+s1}{type}\PY{l+s+s1}{\PYZsq{}}\PY{p}{]}\PY{o}{=}\PY{l+m+mi}{1}
\PY{n}{white\PYZus{}wine}\PY{p}{[}\PY{l+s+s1}{\PYZsq{}}\PY{l+s+s1}{type}\PY{l+s+s1}{\PYZsq{}}\PY{p}{]}\PY{o}{=}\PY{l+m+mi}{0}
\PY{n}{wines}\PY{o}{=}\PY{n}{red\PYZus{}wine}\PY{o}{.}\PY{n}{append}\PY{p}{(}\PY{n}{white\PYZus{}wine}\PY{p}{,}\PY{n}{ignore\PYZus{}index}\PY{o}{=}\PY{k+kc}{True}\PY{p}{)}

\PY{k+kn}{import} \PY{n+nn}{seaborn} \PY{k}{as} \PY{n+nn}{sns}

\PY{n}{corr}\PY{o}{=}\PY{n}{wines}\PY{o}{.}\PY{n}{corr}\PY{p}{(}\PY{p}{)}
\PY{n}{sns}\PY{o}{.}\PY{n}{heatmap}\PY{p}{(}\PY{n}{corr}\PY{p}{,} \PY{n}{xticklabels}\PY{o}{=}\PY{n}{corr}\PY{o}{.}\PY{n}{columns}\PY{o}{.}\PY{n}{values}\PY{p}{,} \PY{n}{yticklabels}\PY{o}{=}\PY{n}{corr}\PY{o}{.}\PY{n}{columns}\PY{o}{.}\PY{n}{values}\PY{p}{)}
\end{Verbatim}
\end{tcolorbox}

            \begin{tcolorbox}[breakable, size=fbox, boxrule=.5pt, pad at break*=1mm, opacityfill=0]
\prompt{Out}{outcolor}{14}{\boxspacing}
\begin{Verbatim}[commandchars=\\\{\}]
<matplotlib.axes.\_subplots.AxesSubplot at 0x1b474810888>
\end{Verbatim}
\end{tcolorbox}
        
    \begin{center}
    \adjustimage{max size={0.9\linewidth}{0.9\paperheight}}{output_18_1.png}
    \end{center}
    { \hspace*{\fill} \\}
    
    From here onwards, we start the proper modeling and building of the
neural network.

    The first step is preparing the training and the test set, in this case
we use a holdout for the test equivalent to 33\% of the total data. The
random\_state is set to 42 for consistent results.

    \begin{tcolorbox}[breakable, size=fbox, boxrule=1pt, pad at break*=1mm,colback=cellbackground, colframe=cellborder]
\prompt{In}{incolor}{15}{\boxspacing}
\begin{Verbatim}[commandchars=\\\{\}]
\PY{k+kn}{from} \PY{n+nn}{sklearn}\PY{n+nn}{.}\PY{n+nn}{model\PYZus{}selection} \PY{k}{import} \PY{n}{train\PYZus{}test\PYZus{}split}
\PY{k+kn}{from} \PY{n+nn}{keras}\PY{n+nn}{.}\PY{n+nn}{utils} \PY{k}{import} \PY{n}{to\PYZus{}categorical}

\PY{n}{X}\PY{o}{=}\PY{n}{wines}\PY{o}{.}\PY{n}{iloc}\PY{p}{[}\PY{p}{:}\PY{p}{,}\PY{l+m+mi}{0}\PY{p}{:}\PY{l+m+mi}{11}\PY{p}{]}
\PY{n}{y}\PY{o}{=}\PY{n}{np}\PY{o}{.}\PY{n}{ravel}\PY{p}{(}\PY{n}{wines}\PY{o}{.}\PY{n}{quality}\PY{p}{)}
\PY{n}{X\PYZus{}train}\PY{p}{,} \PY{n}{X\PYZus{}test}\PY{p}{,} \PY{n}{y\PYZus{}train}\PY{p}{,} \PY{n}{y\PYZus{}test} \PY{o}{=} \PY{n}{train\PYZus{}test\PYZus{}split}\PY{p}{(}\PY{n}{X}\PY{p}{,}\PY{n}{y}\PY{p}{,}\PY{n}{test\PYZus{}size}\PY{o}{=}\PY{l+m+mf}{0.33}\PY{p}{,}\PY{n}{random\PYZus{}state}\PY{o}{=}\PY{l+m+mi}{42}\PY{p}{)}
\end{Verbatim}
\end{tcolorbox}

    \begin{Verbatim}[commandchars=\\\{\}]
Using TensorFlow backend.
    \end{Verbatim}

    Data is normalized in the following section to avoid giving some
attributes bigger importance. This step is important when dealing with
numeric data on several algorithms.

    \begin{tcolorbox}[breakable, size=fbox, boxrule=1pt, pad at break*=1mm,colback=cellbackground, colframe=cellborder]
\prompt{In}{incolor}{16}{\boxspacing}
\begin{Verbatim}[commandchars=\\\{\}]
\PY{k+kn}{from} \PY{n+nn}{sklearn}\PY{n+nn}{.}\PY{n+nn}{preprocessing} \PY{k}{import} \PY{n}{StandardScaler}

\PY{n}{scaler}\PY{o}{=} \PY{n}{StandardScaler}\PY{p}{(}\PY{p}{)}\PY{o}{.}\PY{n}{fit}\PY{p}{(}\PY{n}{X\PYZus{}train}\PY{p}{)} \PY{c+c1}{\PYZsh{}Fit the scaler to the training data}
\PY{n}{X\PYZus{}train} \PY{o}{=} \PY{n}{scaler}\PY{o}{.}\PY{n}{transform}\PY{p}{(}\PY{n}{X\PYZus{}train}\PY{p}{)} \PY{c+c1}{\PYZsh{}Transform the training data}
\PY{n}{X\PYZus{}test} \PY{o}{=} \PY{n}{scaler}\PY{o}{.}\PY{n}{transform}\PY{p}{(}\PY{n}{X\PYZus{}test}\PY{p}{)} \PY{c+c1}{\PYZsh{}Transform the test data, for consistency}
\end{Verbatim}
\end{tcolorbox}

    Build the neural network based on the design and the structure desired.
Must have at the very least one input layer, one hidden layer and one
output layer. Input must have the structure of the data to be received,
in this case we have 12 attributes, which is the amount of neurons we
use. Output is only for the quality of the wine, and quality is in the
range 0:10, so 10 is used. Hidden layers are flexible and there is no
``right'' setup.

    \begin{tcolorbox}[breakable, size=fbox, boxrule=1pt, pad at break*=1mm,colback=cellbackground, colframe=cellborder]
\prompt{In}{incolor}{17}{\boxspacing}
\begin{Verbatim}[commandchars=\\\{\}]
\PY{k+kn}{from} \PY{n+nn}{keras} \PY{k}{import} \PY{n}{Sequential}

\PY{k+kn}{from} \PY{n+nn}{keras}\PY{n+nn}{.}\PY{n+nn}{layers} \PY{k}{import} \PY{n}{Dense}

\PY{n}{model}\PY{o}{=} \PY{n}{Sequential}\PY{p}{(}\PY{p}{)} \PY{c+c1}{\PYZsh{}Define the type of Neural network, Sequential networks move the data in a sequence from input\PYZhy{}\PYZgt{}hidden\PYZhy{}\PYZgt{}output}
\PY{n}{model}\PY{o}{.}\PY{n}{add}\PY{p}{(}\PY{n}{Dense}\PY{p}{(}\PY{l+m+mi}{12}\PY{p}{,}\PY{n}{activation}\PY{o}{=}\PY{l+s+s1}{\PYZsq{}}\PY{l+s+s1}{relu}\PY{l+s+s1}{\PYZsq{}}\PY{p}{,}\PY{n}{input\PYZus{}shape}\PY{o}{=}\PY{p}{(}\PY{l+m+mi}{11}\PY{p}{,}\PY{p}{)}\PY{p}{)}\PY{p}{)} \PY{c+c1}{\PYZsh{}Define the input layer, using \PYZdq{}Dense\PYZdq{} as method, and \PYZsq{}relu\PYZsq{} as activation method}
\PY{n}{model}\PY{o}{.}\PY{n}{add}\PY{p}{(}\PY{n}{Dense}\PY{p}{(}\PY{l+m+mi}{8}\PY{p}{,}\PY{n}{activation}\PY{o}{=}\PY{l+s+s1}{\PYZsq{}}\PY{l+s+s1}{relu}\PY{l+s+s1}{\PYZsq{}}\PY{p}{)}\PY{p}{)} \PY{c+c1}{\PYZsh{}Define hidden layer 1}
\PY{n}{model}\PY{o}{.}\PY{n}{add}\PY{p}{(}\PY{n}{Dense}\PY{p}{(}\PY{l+m+mi}{10}\PY{p}{,}\PY{n}{activation}\PY{o}{=}\PY{l+s+s1}{\PYZsq{}}\PY{l+s+s1}{tanh}\PY{l+s+s1}{\PYZsq{}}\PY{p}{)}\PY{p}{)} \PY{c+c1}{\PYZsh{}Define hidden layer 2}
\PY{n}{model}\PY{o}{.}\PY{n}{add}\PY{p}{(}\PY{n}{Dense}\PY{p}{(}\PY{l+m+mi}{14}\PY{p}{,}\PY{n}{activation}\PY{o}{=}\PY{l+s+s1}{\PYZsq{}}\PY{l+s+s1}{relu}\PY{l+s+s1}{\PYZsq{}}\PY{p}{)}\PY{p}{)} \PY{c+c1}{\PYZsh{}Define hidden layer 3}
\PY{n}{model}\PY{o}{.}\PY{n}{add}\PY{p}{(}\PY{n}{Dense}\PY{p}{(}\PY{l+m+mi}{10}\PY{p}{,}\PY{n}{activation}\PY{o}{=}\PY{l+s+s1}{\PYZsq{}}\PY{l+s+s1}{sigmoid}\PY{l+s+s1}{\PYZsq{}}\PY{p}{)}\PY{p}{)} \PY{c+c1}{\PYZsh{}Define output layer}
\end{Verbatim}
\end{tcolorbox}

    \begin{tcolorbox}[breakable, size=fbox, boxrule=1pt, pad at break*=1mm,colback=cellbackground, colframe=cellborder]
\prompt{In}{incolor}{18}{\boxspacing}
\begin{Verbatim}[commandchars=\\\{\}]
\PY{n}{model}\PY{o}{.}\PY{n}{output\PYZus{}shape}
\end{Verbatim}
\end{tcolorbox}

            \begin{tcolorbox}[breakable, size=fbox, boxrule=.5pt, pad at break*=1mm, opacityfill=0]
\prompt{Out}{outcolor}{18}{\boxspacing}
\begin{Verbatim}[commandchars=\\\{\}]
(None, 10)
\end{Verbatim}
\end{tcolorbox}
        
    \begin{tcolorbox}[breakable, size=fbox, boxrule=1pt, pad at break*=1mm,colback=cellbackground, colframe=cellborder]
\prompt{In}{incolor}{19}{\boxspacing}
\begin{Verbatim}[commandchars=\\\{\}]
\PY{n}{model}\PY{o}{.}\PY{n}{summary}\PY{p}{(}\PY{p}{)}
\end{Verbatim}
\end{tcolorbox}

    \begin{Verbatim}[commandchars=\\\{\}]
Model: "sequential\_1"
\_\_\_\_\_\_\_\_\_\_\_\_\_\_\_\_\_\_\_\_\_\_\_\_\_\_\_\_\_\_\_\_\_\_\_\_\_\_\_\_\_\_\_\_\_\_\_\_\_\_\_\_\_\_\_\_\_\_\_\_\_\_\_\_\_
Layer (type)                 Output Shape              Param \#
=================================================================
dense\_1 (Dense)              (None, 12)                144
\_\_\_\_\_\_\_\_\_\_\_\_\_\_\_\_\_\_\_\_\_\_\_\_\_\_\_\_\_\_\_\_\_\_\_\_\_\_\_\_\_\_\_\_\_\_\_\_\_\_\_\_\_\_\_\_\_\_\_\_\_\_\_\_\_
dense\_2 (Dense)              (None, 8)                 104
\_\_\_\_\_\_\_\_\_\_\_\_\_\_\_\_\_\_\_\_\_\_\_\_\_\_\_\_\_\_\_\_\_\_\_\_\_\_\_\_\_\_\_\_\_\_\_\_\_\_\_\_\_\_\_\_\_\_\_\_\_\_\_\_\_
dense\_3 (Dense)              (None, 10)                90
\_\_\_\_\_\_\_\_\_\_\_\_\_\_\_\_\_\_\_\_\_\_\_\_\_\_\_\_\_\_\_\_\_\_\_\_\_\_\_\_\_\_\_\_\_\_\_\_\_\_\_\_\_\_\_\_\_\_\_\_\_\_\_\_\_
dense\_4 (Dense)              (None, 14)                154
\_\_\_\_\_\_\_\_\_\_\_\_\_\_\_\_\_\_\_\_\_\_\_\_\_\_\_\_\_\_\_\_\_\_\_\_\_\_\_\_\_\_\_\_\_\_\_\_\_\_\_\_\_\_\_\_\_\_\_\_\_\_\_\_\_
dense\_5 (Dense)              (None, 10)                150
=================================================================
Total params: 642
Trainable params: 642
Non-trainable params: 0
\_\_\_\_\_\_\_\_\_\_\_\_\_\_\_\_\_\_\_\_\_\_\_\_\_\_\_\_\_\_\_\_\_\_\_\_\_\_\_\_\_\_\_\_\_\_\_\_\_\_\_\_\_\_\_\_\_\_\_\_\_\_\_\_\_
    \end{Verbatim}

    \begin{tcolorbox}[breakable, size=fbox, boxrule=1pt, pad at break*=1mm,colback=cellbackground, colframe=cellborder]
\prompt{In}{incolor}{20}{\boxspacing}
\begin{Verbatim}[commandchars=\\\{\}]
\PY{n}{model}\PY{o}{.}\PY{n}{get\PYZus{}config}\PY{p}{(}\PY{p}{)}
\end{Verbatim}
\end{tcolorbox}

            \begin{tcolorbox}[breakable, size=fbox, boxrule=.5pt, pad at break*=1mm, opacityfill=0]
\prompt{Out}{outcolor}{20}{\boxspacing}
\begin{Verbatim}[commandchars=\\\{\}]
\{'name': 'sequential\_1',
 'layers': [\{'class\_name': 'Dense',
   'config': \{'name': 'dense\_1',
    'trainable': True,
    'batch\_input\_shape': (None, 11),
    'dtype': 'float32',
    'units': 12,
    'activation': 'relu',
    'use\_bias': True,
    'kernel\_initializer': \{'class\_name': 'VarianceScaling',
     'config': \{'scale': 1.0,
      'mode': 'fan\_avg',
      'distribution': 'uniform',
      'seed': None\}\},
    'bias\_initializer': \{'class\_name': 'Zeros', 'config': \{\}\},
    'kernel\_regularizer': None,
    'bias\_regularizer': None,
    'activity\_regularizer': None,
    'kernel\_constraint': None,
    'bias\_constraint': None\}\},
  \{'class\_name': 'Dense',
   'config': \{'name': 'dense\_2',
    'trainable': True,
    'dtype': 'float32',
    'units': 8,
    'activation': 'relu',
    'use\_bias': True,
    'kernel\_initializer': \{'class\_name': 'VarianceScaling',
     'config': \{'scale': 1.0,
      'mode': 'fan\_avg',
      'distribution': 'uniform',
      'seed': None\}\},
    'bias\_initializer': \{'class\_name': 'Zeros', 'config': \{\}\},
    'kernel\_regularizer': None,
    'bias\_regularizer': None,
    'activity\_regularizer': None,
    'kernel\_constraint': None,
    'bias\_constraint': None\}\},
  \{'class\_name': 'Dense',
   'config': \{'name': 'dense\_3',
    'trainable': True,
    'dtype': 'float32',
    'units': 10,
    'activation': 'tanh',
    'use\_bias': True,
    'kernel\_initializer': \{'class\_name': 'VarianceScaling',
     'config': \{'scale': 1.0,
      'mode': 'fan\_avg',
      'distribution': 'uniform',
      'seed': None\}\},
    'bias\_initializer': \{'class\_name': 'Zeros', 'config': \{\}\},
    'kernel\_regularizer': None,
    'bias\_regularizer': None,
    'activity\_regularizer': None,
    'kernel\_constraint': None,
    'bias\_constraint': None\}\},
  \{'class\_name': 'Dense',
   'config': \{'name': 'dense\_4',
    'trainable': True,
    'dtype': 'float32',
    'units': 14,
    'activation': 'relu',
    'use\_bias': True,
    'kernel\_initializer': \{'class\_name': 'VarianceScaling',
     'config': \{'scale': 1.0,
      'mode': 'fan\_avg',
      'distribution': 'uniform',
      'seed': None\}\},
    'bias\_initializer': \{'class\_name': 'Zeros', 'config': \{\}\},
    'kernel\_regularizer': None,
    'bias\_regularizer': None,
    'activity\_regularizer': None,
    'kernel\_constraint': None,
    'bias\_constraint': None\}\},
  \{'class\_name': 'Dense',
   'config': \{'name': 'dense\_5',
    'trainable': True,
    'dtype': 'float32',
    'units': 10,
    'activation': 'sigmoid',
    'use\_bias': True,
    'kernel\_initializer': \{'class\_name': 'VarianceScaling',
     'config': \{'scale': 1.0,
      'mode': 'fan\_avg',
      'distribution': 'uniform',
      'seed': None\}\},
    'bias\_initializer': \{'class\_name': 'Zeros', 'config': \{\}\},
    'kernel\_regularizer': None,
    'bias\_regularizer': None,
    'activity\_regularizer': None,
    'kernel\_constraint': None,
    'bias\_constraint': None\}\}]\}
\end{Verbatim}
\end{tcolorbox}
        
    \begin{tcolorbox}[breakable, size=fbox, boxrule=1pt, pad at break*=1mm,colback=cellbackground, colframe=cellborder]
\prompt{In}{incolor}{21}{\boxspacing}
\begin{Verbatim}[commandchars=\\\{\}]
\PY{k+kn}{from} \PY{n+nn}{keras}\PY{n+nn}{.}\PY{n+nn}{utils} \PY{k}{import} \PY{n}{to\PYZus{}categorical}
\PY{n}{y\PYZus{}binary} \PY{o}{=} \PY{n}{to\PYZus{}categorical}\PY{p}{(}\PY{n}{y\PYZus{}train}\PY{p}{)}

\PY{n}{model}\PY{o}{.}\PY{n}{get\PYZus{}weights}\PY{p}{(}\PY{p}{)}
\PY{n}{model}\PY{o}{.}\PY{n}{compile}\PY{p}{(}\PY{n}{loss}\PY{o}{=}\PY{l+s+s1}{\PYZsq{}}\PY{l+s+s1}{categorical\PYZus{}crossentropy}\PY{l+s+s1}{\PYZsq{}}\PY{p}{,}\PY{n}{optimizer}\PY{o}{=}\PY{l+s+s1}{\PYZsq{}}\PY{l+s+s1}{adam}\PY{l+s+s1}{\PYZsq{}}\PY{p}{,}\PY{n}{metrics}\PY{o}{=}\PY{p}{[}\PY{l+s+s1}{\PYZsq{}}\PY{l+s+s1}{accuracy}\PY{l+s+s1}{\PYZsq{}}\PY{p}{]}\PY{p}{)}
\PY{n}{model}\PY{o}{.}\PY{n}{fit}\PY{p}{(}\PY{n}{X\PYZus{}train}\PY{p}{,}\PY{n}{y\PYZus{}binary}\PY{p}{,}\PY{n}{epochs}\PY{o}{=}\PY{l+m+mi}{10}\PY{p}{,}\PY{n}{batch\PYZus{}size}\PY{o}{=}\PY{l+m+mi}{1}\PY{p}{,}\PY{n}{verbose}\PY{o}{=}\PY{l+m+mi}{1}\PY{p}{)}
\end{Verbatim}
\end{tcolorbox}

    \begin{Verbatim}[commandchars=\\\{\}]
Epoch 1/10
4352/4352 [==============================] - 8s 2ms/step - loss: 1.1954 -
accuracy: 0.4892
Epoch 2/10
4352/4352 [==============================] - 9s 2ms/step - loss: 1.0770 -
accuracy: 0.5448
Epoch 3/10
4352/4352 [==============================] - 9s 2ms/step - loss: 1.0590 -
accuracy: 0.5556
Epoch 4/10
4352/4352 [==============================] - 9s 2ms/step - loss: 1.0508 -
accuracy: 0.5604
Epoch 5/10
4352/4352 [==============================] - 9s 2ms/step - loss: 1.0419 -
accuracy: 0.5565
Epoch 6/10
4352/4352 [==============================] - 8s 2ms/step - loss: 1.0341 -
accuracy: 0.5682
Epoch 7/10
4352/4352 [==============================] - 8s 2ms/step - loss: 1.0276 -
accuracy: 0.5655
Epoch 8/10
4352/4352 [==============================] - 8s 2ms/step - loss: 1.0203 -
accuracy: 0.5669
Epoch 9/10
4352/4352 [==============================] - 8s 2ms/step - loss: 1.0151 -
accuracy: 0.5699
Epoch 10/10
4352/4352 [==============================] - 8s 2ms/step - loss: 1.0116 -
accuracy: 0.5761
    \end{Verbatim}

            \begin{tcolorbox}[breakable, size=fbox, boxrule=.5pt, pad at break*=1mm, opacityfill=0]
\prompt{Out}{outcolor}{21}{\boxspacing}
\begin{Verbatim}[commandchars=\\\{\}]
<keras.callbacks.callbacks.History at 0x1b47b3934c8>
\end{Verbatim}
\end{tcolorbox}
        
    \begin{tcolorbox}[breakable, size=fbox, boxrule=1pt, pad at break*=1mm,colback=cellbackground, colframe=cellborder]
\prompt{In}{incolor}{22}{\boxspacing}
\begin{Verbatim}[commandchars=\\\{\}]
\PY{n}{y\PYZus{}pred}\PY{o}{=}\PY{n}{model}\PY{o}{.}\PY{n}{predict\PYZus{}classes}\PY{p}{(}\PY{n}{X\PYZus{}test}\PY{p}{)}

\PY{c+c1}{\PYZsh{}np.round(y\PYZus{}pred,1)}
\end{Verbatim}
\end{tcolorbox}

    \begin{tcolorbox}[breakable, size=fbox, boxrule=1pt, pad at break*=1mm,colback=cellbackground, colframe=cellborder]
\prompt{In}{incolor}{23}{\boxspacing}
\begin{Verbatim}[commandchars=\\\{\}]
\PY{n}{y\PYZus{}pred}\PY{p}{[}\PY{p}{:}\PY{l+m+mi}{10}\PY{p}{]}
\end{Verbatim}
\end{tcolorbox}

            \begin{tcolorbox}[breakable, size=fbox, boxrule=.5pt, pad at break*=1mm, opacityfill=0]
\prompt{Out}{outcolor}{23}{\boxspacing}
\begin{Verbatim}[commandchars=\\\{\}]
array([6, 5, 7, 5, 5, 6, 5, 6, 5, 6], dtype=int64)
\end{Verbatim}
\end{tcolorbox}
        
    \begin{tcolorbox}[breakable, size=fbox, boxrule=1pt, pad at break*=1mm,colback=cellbackground, colframe=cellborder]
\prompt{In}{incolor}{24}{\boxspacing}
\begin{Verbatim}[commandchars=\\\{\}]
\PY{n}{y\PYZus{}test}\PY{p}{[}\PY{p}{:}\PY{l+m+mi}{10}\PY{p}{]}
\end{Verbatim}
\end{tcolorbox}

            \begin{tcolorbox}[breakable, size=fbox, boxrule=.5pt, pad at break*=1mm, opacityfill=0]
\prompt{Out}{outcolor}{24}{\boxspacing}
\begin{Verbatim}[commandchars=\\\{\}]
array([8, 5, 7, 6, 6, 6, 5, 6, 5, 7], dtype=int64)
\end{Verbatim}
\end{tcolorbox}
        
    \begin{tcolorbox}[breakable, size=fbox, boxrule=1pt, pad at break*=1mm,colback=cellbackground, colframe=cellborder]
\prompt{In}{incolor}{25}{\boxspacing}
\begin{Verbatim}[commandchars=\\\{\}]
\PY{n}{y\PYZus{}binary\PYZus{}test} \PY{o}{=} \PY{n}{to\PYZus{}categorical}\PY{p}{(}\PY{n}{y\PYZus{}test}\PY{p}{)}
\PY{n}{score}\PY{o}{=}\PY{n}{model}\PY{o}{.}\PY{n}{evaluate}\PY{p}{(}\PY{n}{X\PYZus{}test}\PY{p}{,}\PY{n}{y\PYZus{}binary\PYZus{}test}\PY{p}{,}\PY{n}{verbose}\PY{o}{=}\PY{l+m+mi}{1}\PY{p}{)}
\PY{n+nb}{print}\PY{p}{(}\PY{n}{score}\PY{p}{)}
\end{Verbatim}
\end{tcolorbox}

    \begin{Verbatim}[commandchars=\\\{\}]
2145/2145 [==============================] - 0s 70us/step
[1.0448312323688072, 0.5524475574493408]
    \end{Verbatim}

    \begin{tcolorbox}[breakable, size=fbox, boxrule=1pt, pad at break*=1mm,colback=cellbackground, colframe=cellborder]
\prompt{In}{incolor}{26}{\boxspacing}
\begin{Verbatim}[commandchars=\\\{\}]
\PY{k+kn}{from} \PY{n+nn}{sklearn}\PY{n+nn}{.}\PY{n+nn}{metrics} \PY{k}{import} \PY{n}{confusion\PYZus{}matrix}\PY{p}{,} \PY{n}{precision\PYZus{}score}\PY{p}{,} \PY{n}{recall\PYZus{}score}\PY{p}{,} \PY{n}{f1\PYZus{}score}\PY{p}{,} \PY{n}{cohen\PYZus{}kappa\PYZus{}score}
\PY{n}{confusion\PYZus{}matrix}\PY{p}{(}\PY{n}{y\PYZus{}test}\PY{p}{,}\PY{n}{y\PYZus{}pred}\PY{p}{)}
\end{Verbatim}
\end{tcolorbox}

            \begin{tcolorbox}[breakable, size=fbox, boxrule=.5pt, pad at break*=1mm, opacityfill=0]
\prompt{Out}{outcolor}{26}{\boxspacing}
\begin{Verbatim}[commandchars=\\\{\}]
array([[  0,   0,   5,   4,   1,   0,   0],
       [  0,   0,  50,  26,   0,   0,   0],
       [  0,   0, 449, 211,  12,   0,   0],
       [  0,   0, 274, 618,  85,   0,   0],
       [  0,   0,   8, 226, 118,   0,   0],
       [  0,   0,   0,  27,  30,   0,   0],
       [  0,   0,   0,   0,   1,   0,   0]], dtype=int64)
\end{Verbatim}
\end{tcolorbox}
        
    \begin{tcolorbox}[breakable, size=fbox, boxrule=1pt, pad at break*=1mm,colback=cellbackground, colframe=cellborder]
\prompt{In}{incolor}{27}{\boxspacing}
\begin{Verbatim}[commandchars=\\\{\}]
\PY{n}{precision\PYZus{}score}\PY{p}{(}\PY{n}{y\PYZus{}test}\PY{p}{,}\PY{n}{y\PYZus{}pred}\PY{p}{,}\PY{n}{average}\PY{o}{=}\PY{l+s+s2}{\PYZdq{}}\PY{l+s+s2}{weighted}\PY{l+s+s2}{\PYZdq{}}\PY{p}{)}
\end{Verbatim}
\end{tcolorbox}

    \begin{Verbatim}[commandchars=\\\{\}]
D:\textbackslash{}Anaconda3\textbackslash{}lib\textbackslash{}site-packages\textbackslash{}sklearn\textbackslash{}metrics\textbackslash{}classification.py:1437:
UndefinedMetricWarning: Precision is ill-defined and being set to 0.0 in labels
with no predicted samples.
  'precision', 'predicted', average, warn\_for)
    \end{Verbatim}

            \begin{tcolorbox}[breakable, size=fbox, boxrule=.5pt, pad at break*=1mm, opacityfill=0]
\prompt{Out}{outcolor}{27}{\boxspacing}
\begin{Verbatim}[commandchars=\\\{\}]
0.5104954906750544
\end{Verbatim}
\end{tcolorbox}
        
    \begin{tcolorbox}[breakable, size=fbox, boxrule=1pt, pad at break*=1mm,colback=cellbackground, colframe=cellborder]
\prompt{In}{incolor}{28}{\boxspacing}
\begin{Verbatim}[commandchars=\\\{\}]
\PY{n}{recall\PYZus{}score}\PY{p}{(}\PY{n}{y\PYZus{}test}\PY{p}{,}\PY{n}{y\PYZus{}pred}\PY{p}{,}\PY{n}{average}\PY{o}{=}\PY{l+s+s2}{\PYZdq{}}\PY{l+s+s2}{weighted}\PY{l+s+s2}{\PYZdq{}}\PY{p}{)}
\end{Verbatim}
\end{tcolorbox}

            \begin{tcolorbox}[breakable, size=fbox, boxrule=.5pt, pad at break*=1mm, opacityfill=0]
\prompt{Out}{outcolor}{28}{\boxspacing}
\begin{Verbatim}[commandchars=\\\{\}]
0.5524475524475524
\end{Verbatim}
\end{tcolorbox}
        
    \begin{tcolorbox}[breakable, size=fbox, boxrule=1pt, pad at break*=1mm,colback=cellbackground, colframe=cellborder]
\prompt{In}{incolor}{29}{\boxspacing}
\begin{Verbatim}[commandchars=\\\{\}]
\PY{n}{f1\PYZus{}score}\PY{p}{(}\PY{n}{y\PYZus{}test}\PY{p}{,}\PY{n}{y\PYZus{}pred}\PY{p}{,}\PY{n}{average}\PY{o}{=}\PY{l+s+s2}{\PYZdq{}}\PY{l+s+s2}{weighted}\PY{l+s+s2}{\PYZdq{}}\PY{p}{)}
\end{Verbatim}
\end{tcolorbox}

    \begin{Verbatim}[commandchars=\\\{\}]
D:\textbackslash{}Anaconda3\textbackslash{}lib\textbackslash{}site-packages\textbackslash{}sklearn\textbackslash{}metrics\textbackslash{}classification.py:1437:
UndefinedMetricWarning: F-score is ill-defined and being set to 0.0 in labels
with no predicted samples.
  'precision', 'predicted', average, warn\_for)
    \end{Verbatim}

            \begin{tcolorbox}[breakable, size=fbox, boxrule=.5pt, pad at break*=1mm, opacityfill=0]
\prompt{Out}{outcolor}{29}{\boxspacing}
\begin{Verbatim}[commandchars=\\\{\}]
0.5271047594271203
\end{Verbatim}
\end{tcolorbox}
        
    \begin{tcolorbox}[breakable, size=fbox, boxrule=1pt, pad at break*=1mm,colback=cellbackground, colframe=cellborder]
\prompt{In}{incolor}{30}{\boxspacing}
\begin{Verbatim}[commandchars=\\\{\}]
\PY{n}{cohen\PYZus{}kappa\PYZus{}score}\PY{p}{(}\PY{n}{y\PYZus{}test}\PY{p}{,}\PY{n}{y\PYZus{}pred}\PY{p}{)}
\end{Verbatim}
\end{tcolorbox}

            \begin{tcolorbox}[breakable, size=fbox, boxrule=.5pt, pad at break*=1mm, opacityfill=0]
\prompt{Out}{outcolor}{30}{\boxspacing}
\begin{Verbatim}[commandchars=\\\{\}]
0.28980001483032225
\end{Verbatim}
\end{tcolorbox}
        
    Conclusiones\textgreater{}

    Ya casi 3 horas después de haber comenzado, creo que tengo la idea
básica aprendida, aunque tuve que recurrir bastante a google para lograr
definir el tipo de red y predicción como multi-clase. En el caso de
ejemplo inicial con el tipo de vino, una comparación binaria tradicional
la precisión era excelente, pero ya al querer determinar la calidad del
vino se complicó más de lo que hubiera querido. Tampoco estoy del todo
familiarizado con Python debido a gran parte del máster ser en R, y me
confundo en hacer operaciones de mezcla y separación, selección de
atributos y otros, además de que los modelos tienen formato diferente en
python, y hay que usar mucho numpy. Forzar capas ocultas intermedias
para probar los puntos inferiores mencionados y cambiar de función y
usar tanh afectó la precisión final, en fin, es un ejercicio de práctica
para entender los fundamentos básicos así que tampoco espero buscar
99.99\% de precisión, pero solo 57.61\% máximo me hace preguntarme si
realmente todo se hizo como se debía.

    La parte de los parámetros de las capas se me quedó un poco difusa hasta
que fui a Google, realmente los nombres usados en la función no son
exactamente claros para alguien aprendiendo, incluso llegué a pensar que
el 1 del output layer era porque se daba una única solución y no porque
era en el rango 0,1. Luego de estar dando vueltas al código fue que
logré entender que son necesarias múltiples neuronas en la capa de
salida para dar valores superiores a 1, y fue que pude hacer la parte de
clasificación multiclase.


    % Add a bibliography block to the postdoc
    
    
    
\end{document}
